%% Generated by Sphinx.
\def\sphinxdocclass{report}
\documentclass[letterpaper,10pt,english]{sphinxmanual}
\ifdefined\pdfpxdimen
   \let\sphinxpxdimen\pdfpxdimen\else\newdimen\sphinxpxdimen
\fi \sphinxpxdimen=.75bp\relax
\ifdefined\pdfimageresolution
    \pdfimageresolution= \numexpr \dimexpr1in\relax/\sphinxpxdimen\relax
\fi
%% let collapsable pdf bookmarks panel have high depth per default
\PassOptionsToPackage{bookmarksdepth=5}{hyperref}

\PassOptionsToPackage{warn}{textcomp}
\usepackage[utf8]{inputenc}
\ifdefined\DeclareUnicodeCharacter
% support both utf8 and utf8x syntaxes
  \ifdefined\DeclareUnicodeCharacterAsOptional
    \def\sphinxDUC#1{\DeclareUnicodeCharacter{"#1}}
  \else
    \let\sphinxDUC\DeclareUnicodeCharacter
  \fi
  \sphinxDUC{00A0}{\nobreakspace}
  \sphinxDUC{2500}{\sphinxunichar{2500}}
  \sphinxDUC{2502}{\sphinxunichar{2502}}
  \sphinxDUC{2514}{\sphinxunichar{2514}}
  \sphinxDUC{251C}{\sphinxunichar{251C}}
  \sphinxDUC{2572}{\textbackslash}
\fi
\usepackage{cmap}
\usepackage[T1]{fontenc}
\usepackage{amsmath,amssymb,amstext}
\usepackage{babel}



\usepackage{tgtermes}
\usepackage{tgheros}
\renewcommand{\ttdefault}{txtt}



\usepackage[Bjarne]{fncychap}
\usepackage{sphinx}

\fvset{fontsize=auto}
\usepackage{geometry}


% Include hyperref last.
\usepackage{hyperref}
% Fix anchor placement for figures with captions.
\usepackage{hypcap}% it must be loaded after hyperref.
% Set up styles of URL: it should be placed after hyperref.
\urlstyle{same}


\usepackage{sphinxmessages}
\setcounter{tocdepth}{0}



\title{DREAMIN\textquotesingle{}SG Integrated Simulation Platform}
\date{Jul 28, 2021}
\release{}
\author{Srijith Balakrishnan}
\newcommand{\sphinxlogo}{\vbox{}}
\renewcommand{\releasename}{}
\makeindex
\begin{document}

\pagestyle{empty}
\sphinxmaketitle
\pagestyle{plain}
\sphinxtableofcontents
\pagestyle{normal}
\phantomsection\label{\detokenize{index::doc}}



\chapter{Introduction}
\label{\detokenize{intro:introduction}}\label{\detokenize{intro::doc}}
\sphinxAtStartPar
TThis package is developed as part of the Disaster REsilience Assessment,
Modelling, and INno\sphinxhyphen{}vation Singapore (DREAMIN’SG) project at the Future
Resilient Systems of the Singapore\sphinxhyphen{}ETH Centre. The DREAMIN’SG project is
funded by the National Research Foundation, Singapore under the Intra\sphinxhyphen{}CREATE
grant program. In the wake of increasing threats posed by climate change
on urban infrastructure {[}Nissen and Ulbrich (2017){]}, this project aims at
studying the effects of policy interventions and network characteristics on
the resilience of urban infrastructure networks. Given the critical nature of
the above infrastructure systems, their interdependencies need to be
considered in pre\sphinxhyphen{} and post\sphinxhyphen{}disaster resilience actions.

\sphinxAtStartPar
In specific, the DREAMIN’SG project envisages to build a methodology to assess and predict
the resilience of urban infrastructure systems and propose new pathways to
develop innovative technologies and services for its improvement. The urban
infrastructure system is modeled as an interdependent power\sphinxhyphen{}, water\sphinxhyphen{}, and
transportation network that interact with each other before, during and
after a disaster. The researchers are developing an integrated simulation
model to study the performance of the interdependent infrastructure network
under various disruption and recov\sphinxhyphen{}ery scenarios. Based on the
simulation\sphinxhyphen{}generated datasets, machine learning algorithms would be implemented
to understand the causal relationship between topological and policy\sphinxhyphen{}related
interventions and disaster risks. Building on the understanding of success
features that make an urban system resilient, Design Science approaches will
be used to develop new solutions that mitigate the consequences of
disruptions and accommodate constraints in the analysed case studies.
The results of the research will support local governments and system managers
to improve infrastructure resilience against weather\sphinxhyphen{}related disruptions.
The methodology adopted in the study is presented in Fig. 1.

\begin{figure}[htbp]
\centering
\capstart

\noindent\sphinxincludegraphics[scale=0.5]{{methodology}.PNG}
\caption{Figure 1. Methodological framework of DREAMIN’SG project}\label{\detokenize{intro:id1}}\end{figure}

\sphinxAtStartPar
The steps of the project are summarized as follows:
\begin{enumerate}
\sphinxsetlistlabels{\arabic}{enumi}{enumii}{}{.}%
\item {} 
\sphinxAtStartPar
Multiple scenarios are generated by considering different disruptions, network models, technological constraints and system configurations.

\item {} 
\sphinxAtStartPar
A simulation model is created for the interdependent power grid, water distribution system, and road transportation system.

\item {} 
\sphinxAtStartPar
Resilience is assessed based on the simulated performance of the three systems.

\item {} 
\sphinxAtStartPar
An interpretable machine learning algorithm is implemented to analyze the scenarios and extract information related to key system features that influence resilience.

\item {} 
\sphinxAtStartPar
The identified system features inform the design of new services, technologies, and prod\sphinxhyphen{}ucts that are able to simultaneously enhance resilience and accommodate the technological constraints.

\end{enumerate}

\sphinxAtStartPar
For further information and updates on the project, please visit the
\sphinxhref{https://frs.ethz.ch/research/projects/dreamin\_sg.html}{DREAMIN’SG webpage}.
In the rest of the documentation, the details of the interdependent infrastructure
simulation platform,including its modeling, installation, and usage are
discussed.


\chapter{Integrated Simulation Platform}
\label{\detokenize{model:integrated-simulation-platform}}\label{\detokenize{model::doc}}
\sphinxAtStartPar
The integrated simulation model has been developed as a Python\sphinxhyphen{}based package consisting of modules for simulation of system\sphinxhyphen{} and network\sphinxhyphen{}level
cascading effects resulting from component failures. The overall structure of the integrated simulation platform is illustrated in Figure 2.

\begin{figure}[htbp]
\centering
\capstart

\noindent\sphinxincludegraphics[scale=0.5]{{structure}.PNG}
\caption{Figure 2. DREAMIN’SG integrated simulation platform structure}\label{\detokenize{model:id1}}\end{figure}

\sphinxAtStartPar
The model is capable of initializing disaster scenarios in interdependent power, water and transportation
networks and evaluating resilience strategies by generating operational performance curves (Figure 3).
The resilience strategies that can be tested include pre\sphinxhyphen{}disaster interventions, such as system redundancy
enhancements and post\sphinxhyphen{}disaster recovery optimization.

\begin{figure}[htbp]
\centering
\capstart

\noindent\sphinxincludegraphics[scale=0.6]{{perf_curves}.PNG}
\caption{Figure 3. Implementation of the simulation platform to generate performance curves}\label{\detokenize{model:id2}}\end{figure}

\sphinxAtStartPar
The model is developed by integrating existing flow\sphinxhyphen{}based water, power and transportation network models.
The whole model can be divided into three broad modules, namely, the integrated infrastructure network module, network recovery module, and the
recovery optimization module.


\section{Integrated infrastructure network}
\label{\detokenize{model:integrated-infrastructure-network}}
\sphinxAtStartPar
This module houses the three infrastructure network models which are used to simulate power\sphinxhyphen{},
water\sphinxhyphen{} and transportation networks independently. These are developed using existing Python\sphinxhyphen{}based packages and the details are presented in
Table 2. The module also consists of an interdependency sub\sphinxhyphen{}module which serves as an interface between infrastructure network pairs.
Currently the following dependencies are considered in the interdependent simulation platform.
\begin{enumerate}
\sphinxsetlistlabels{\alph}{enumi}{enumii}{}{.}%
\item {} 
\sphinxAtStartPar
Power\sphinxhyphen{}water dependencies include dependency of water pumps on electric motors and generators on reservoirs.

\item {} 
\sphinxAtStartPar
Dependencies also exist between traffic networks and the other two infrastructure models, as the former provides access to the latter, which is critical during the recovery phase. The module also stores the details of the states of all network components, including their operational status after a disaster.

\end{enumerate}


\section{Network recovery}
\label{\detokenize{model:network-recovery}}
\sphinxAtStartPar
The recovery module consists of functions to develop an event table to schedule disruptive events and restoration actions after a disaster
is initiated in the model. The simulation platform uses this table as a reference to modify the operational status of network components
during a simulation, so that the consequences of disaster events and repair actions are reflected while simulating network performance. The
recovery module also stores the details such as the number of repair crew for every infrastructure network, their initial locations, etc.


\section{Recovery optimization}
\label{\detokenize{model:recovery-optimization}}
\sphinxAtStartPar
This module determines the order in which the repair actions are carried out. Currently, the approach of the optimization module leverages
on the methodology of Model Predictive Control (MPC). In this approach, first, out of \sphinxstyleemphasis{N} repair steps, the solution considering only \sphinxstyleemphasis{k} steps,
called the prediction horizon, is computed. Next, the first step of the obtained solution is applied to the system and then the process is
being repeated for the remaining \sphinxstyleemphasis{N\sphinxhyphen{}1} components until all components have been scheduled for repair. In the context of the integrated infrastructure
simulation, the optimizer module evaluates repair sequences of the length of the prediction horizon for each infrastructure (assuming that each of
the infrastructure has a separate recovery crew) based on a resilience metric. In the model, the integral loss of service (ILOS) is used as the
resilience metric. The ILOS is calculated as follows:
\begin{equation*}
\begin{split}ILOS = w_{P}\sum_{k}DPower\times dt_{P}(k)+w_{W}\sum_{k}DWater\times dt_{w}(k)+w_{T}\sum_{k}DTransport\times dt_{T}(k)\end{split}
\end{equation*}
\sphinxAtStartPar
where, \(DPower\), \(DWater\) and \(DTransport\) are the respective demands not served, \(dt_{P}(k)\), \(dt_{W}(k)\) and \(dt_{T}(k)\)
the repair\sphinxhyphen{}time specific time steps between the repair actions and wp,ww,wTthe weights. The optimal repair sequence is found by minimizing the ILOS. At this stage
the optimal repair action in each prediction horizon is computed using a brute\sphinxhyphen{}force approach where the ILOS is evaluated for each of the repair sequences.

\sphinxAtStartPar
Currently, the network data, interdependency data and infrastructure disruption data are to be manually fed into the model to run the network simulations.
However, efforts are being made to include separate modules for network generation and hazard initiation in the simulation platform, to enhance the scope of the model.
The following improvements will also be made to the model:
\begin{enumerate}
\sphinxsetlistlabels{\alph}{enumi}{enumii}{}{.}%
\item {} 
\sphinxAtStartPar
Realistic policies for network recovery.

\item {} 
\sphinxAtStartPar
Additional interdependencies.

\end{enumerate}


\chapter{Data requirements}
\label{\detokenize{data_req:data-requirements}}\label{\detokenize{data_req::doc}}

\chapter{Installation}
\label{\detokenize{installation:installation}}\label{\detokenize{installation::doc}}

\section{Stable release}
\label{\detokenize{installation:stable-release}}
\sphinxAtStartPar
To install dreaminsg\sphinxhyphen{}integrated\sphinxhyphen{}model, run this command in your terminal:

\begin{sphinxVerbatim}[commandchars=\\\{\}]
\PYG{g+gp}{\PYGZdl{} }pip install dreaminsg\PYGZus{}integrated\PYGZus{}model
\end{sphinxVerbatim}

\sphinxAtStartPar
This is the preferred method to install dreaminsg\sphinxhyphen{}integrated\sphinxhyphen{}model, as it will always install the most recent stable release.

\sphinxAtStartPar
If you don’t have \sphinxhref{https://pip.pypa.io}{pip} installed, this \sphinxhref{http://docs.python-guide.org/en/latest/starting/installation/}{Python installation guide} can guide
you through the process.


\section{From sources}
\label{\detokenize{installation:from-sources}}
\sphinxAtStartPar
The sources for dreaminsg\sphinxhyphen{}integrated\sphinxhyphen{}model can be downloaded from the \sphinxhref{https://github.com/srijithabalakrishnan/dreaminsg\_integrated\_model}{Github repo}.

\sphinxAtStartPar
You can either clone the public repository:

\begin{sphinxVerbatim}[commandchars=\\\{\}]
\PYG{g+gp}{\PYGZdl{} }git clone git://github.com/srijithabalakrishnan/dreaminsg\PYGZus{}integrated\PYGZus{}model
\end{sphinxVerbatim}

\sphinxAtStartPar
Or download the \sphinxhref{https://github.com/srijithabalakrishnan/dreaminsg\_integrated\_model/tarball/master}{tarball}:

\begin{sphinxVerbatim}[commandchars=\\\{\}]
\PYG{g+gp}{\PYGZdl{} }curl \PYGZhy{}OJL https://github.com/srijithabalakrishnan/dreaminsg\PYGZus{}integrated\PYGZus{}model/tarball/master
\end{sphinxVerbatim}

\sphinxAtStartPar
Once you have a copy of the source, you can install it with:

\begin{sphinxVerbatim}[commandchars=\\\{\}]
\PYG{g+gp}{\PYGZdl{} }python setup.py install
\end{sphinxVerbatim}

\sphinxAtStartPar
You may need to create a new Python environment that has all required packages and dependencies installed before start using the package.
Run the following comment.

\begin{sphinxVerbatim}[commandchars=\\\{\}]
\PYG{g+gp}{\PYGZdl{} }conda env create \PYGZhy{}\PYGZhy{}name ENV\PYGZus{}NAME \PYGZhy{}\PYGZhy{}file\PYG{o}{=}environment.yml
\end{sphinxVerbatim}


\chapter{Getting started}
\label{\detokenize{use:getting-started}}\label{\detokenize{use::doc}}

\chapter{Support}
\label{\detokenize{support:support}}\label{\detokenize{support::doc}}
\sphinxAtStartPar
The model is still in development stage. To report any issues in the model,
suggest changes, or obtain support with installation or use, please write to
\sphinxhref{mailto:srijith.balakrishnan@sec.ethz.ch}{srijith.balakrishnan{[}at{]}sec.ethz.ch}


\chapter{API documentation}
\label{\detokenize{apidoc:api-documentation}}\label{\detokenize{apidoc::doc}}

\section{dreaminsg\_integrated\_model.main module}
\label{\detokenize{apidoc:module-dreaminsg_integrated_model.main}}\label{\detokenize{apidoc:dreaminsg-integrated-model-main-module}}\index{module@\spxentry{module}!dreaminsg\_integrated\_model.main@\spxentry{dreaminsg\_integrated\_model.main}}\index{dreaminsg\_integrated\_model.main@\spxentry{dreaminsg\_integrated\_model.main}!module@\spxentry{module}}
\sphinxAtStartPar
This is the main module of the integrated infrastructure model where the simulations are performed.
\index{main() (in module dreaminsg\_integrated\_model.main)@\spxentry{main()}\spxextra{in module dreaminsg\_integrated\_model.main}}

\begin{fulllineitems}
\phantomsection\label{\detokenize{apidoc:dreaminsg_integrated_model.main.main}}\pysiglinewithargsret{\sphinxbfcode{\sphinxupquote{main}}}{}{}
\sphinxAtStartPar
This is the main function that contains the whole simulation workflow.

\end{fulllineitems}



\section{dreaminsg\_integrated\_model.src package}
\label{\detokenize{apidoc:dreaminsg-integrated-model-src-package}}

\subsection{dreaminsg\_integrated\_model.src.network\_recovery}
\label{\detokenize{apidoc:dreaminsg-integrated-model-src-network-recovery}}\phantomsection\label{\detokenize{apidoc:module-dreaminsg_integrated_model.src.network_recovery}}\index{module@\spxentry{module}!dreaminsg\_integrated\_model.src.network\_recovery@\spxentry{dreaminsg\_integrated\_model.src.network\_recovery}}\index{dreaminsg\_integrated\_model.src.network\_recovery@\spxentry{dreaminsg\_integrated\_model.src.network\_recovery}!module@\spxentry{module}}
\sphinxAtStartPar
Functions to generate and save disruptive scenarios.
\index{NetworkRecovery (class in dreaminsg\_integrated\_model.src.network\_recovery)@\spxentry{NetworkRecovery}\spxextra{class in dreaminsg\_integrated\_model.src.network\_recovery}}

\begin{fulllineitems}
\phantomsection\label{\detokenize{apidoc:dreaminsg_integrated_model.src.network_recovery.NetworkRecovery}}\pysiglinewithargsret{\sphinxbfcode{\sphinxupquote{class }}\sphinxbfcode{\sphinxupquote{NetworkRecovery}}}{\emph{\DUrole{n}{network}}, \emph{\DUrole{n}{sim\_step}}}{}
\sphinxAtStartPar
Bases: \sphinxcode{\sphinxupquote{object}}

\sphinxAtStartPar
Generate a disaster and recovery object for storing simulation\sphinxhyphen{}related information and settings.
\index{fail\_transpo\_link() (NetworkRecovery method)@\spxentry{fail\_transpo\_link()}\spxextra{NetworkRecovery method}}

\begin{fulllineitems}
\phantomsection\label{\detokenize{apidoc:dreaminsg_integrated_model.src.network_recovery.NetworkRecovery.fail_transpo_link}}\pysiglinewithargsret{\sphinxbfcode{\sphinxupquote{fail\_transpo\_link}}}{\emph{\DUrole{n}{link\_compon}}}{}
\sphinxAtStartPar
Fails the given transportation link by changing the free\sphinxhyphen{}flow travel time to a very large value.
\begin{description}
\item[{Args:}] \leavevmode
\sphinxAtStartPar
link\_compon (string): Name of the transportation link.

\end{description}

\end{fulllineitems}

\index{get\_event\_table() (NetworkRecovery method)@\spxentry{get\_event\_table()}\spxextra{NetworkRecovery method}}

\begin{fulllineitems}
\phantomsection\label{\detokenize{apidoc:dreaminsg_integrated_model.src.network_recovery.NetworkRecovery.get_event_table}}\pysiglinewithargsret{\sphinxbfcode{\sphinxupquote{get\_event\_table}}}{}{}
\sphinxAtStartPar
Returns the event table.

\end{fulllineitems}

\index{initiate\_next\_recov\_scheduled() (NetworkRecovery method)@\spxentry{initiate\_next\_recov\_scheduled()}\spxextra{NetworkRecovery method}}

\begin{fulllineitems}
\phantomsection\label{\detokenize{apidoc:dreaminsg_integrated_model.src.network_recovery.NetworkRecovery.initiate_next_recov_scheduled}}\pysiglinewithargsret{\sphinxbfcode{\sphinxupquote{initiate\_next\_recov\_scheduled}}}{}{}
\sphinxAtStartPar
Flag to identify when the crew must stop at a point and schedule next recovery

\end{fulllineitems}

\index{reset\_networks() (NetworkRecovery method)@\spxentry{reset\_networks()}\spxextra{NetworkRecovery method}}

\begin{fulllineitems}
\phantomsection\label{\detokenize{apidoc:dreaminsg_integrated_model.src.network_recovery.NetworkRecovery.reset_networks}}\pysiglinewithargsret{\sphinxbfcode{\sphinxupquote{reset\_networks}}}{}{}
\sphinxAtStartPar
Resets the IntegratedNetwork object within NetworkRecovery object.

\end{fulllineitems}

\index{restore\_transpo\_link() (NetworkRecovery method)@\spxentry{restore\_transpo\_link()}\spxextra{NetworkRecovery method}}

\begin{fulllineitems}
\phantomsection\label{\detokenize{apidoc:dreaminsg_integrated_model.src.network_recovery.NetworkRecovery.restore_transpo_link}}\pysiglinewithargsret{\sphinxbfcode{\sphinxupquote{restore\_transpo\_link}}}{\emph{\DUrole{n}{link\_compon}}}{}
\sphinxAtStartPar
Restores the disrupted transportation link by changing the free flow travel time to the original value.
\begin{description}
\item[{Args:}] \leavevmode
\sphinxAtStartPar
link\_compon (string): Name of the transportation link.

\end{description}

\end{fulllineitems}

\index{schedule\_recovery() (NetworkRecovery method)@\spxentry{schedule\_recovery()}\spxextra{NetworkRecovery method}}

\begin{fulllineitems}
\phantomsection\label{\detokenize{apidoc:dreaminsg_integrated_model.src.network_recovery.NetworkRecovery.schedule_recovery}}\pysiglinewithargsret{\sphinxbfcode{\sphinxupquote{schedule\_recovery}}}{\emph{\DUrole{n}{repair\_order}}}{}
\sphinxAtStartPar
Generates the unexpanded event table consisting of disruptions and repair actions.
\begin{quote}\begin{description}
\item[{Parameters}] \leavevmode
\sphinxAtStartPar
\sphinxstyleliteralstrong{\sphinxupquote{repair\_order}} (\sphinxstyleliteralemphasis{\sphinxupquote{list of strings.}}) \textendash{} The repair order considered in the current simulation.

\end{description}\end{quote}

\end{fulllineitems}

\index{set\_initial\_crew\_start() (NetworkRecovery method)@\spxentry{set\_initial\_crew\_start()}\spxextra{NetworkRecovery method}}

\begin{fulllineitems}
\phantomsection\label{\detokenize{apidoc:dreaminsg_integrated_model.src.network_recovery.NetworkRecovery.set_initial_crew_start}}\pysiglinewithargsret{\sphinxbfcode{\sphinxupquote{set\_initial\_crew\_start}}}{\emph{\DUrole{n}{repair\_order}}}{}
\sphinxAtStartPar
Sets the initial start times at which the respective infrastructure crews start from their locations post\sphinxhyphen{}disaster.
\begin{quote}\begin{description}
\item[{Parameters}] \leavevmode
\sphinxAtStartPar
\sphinxstyleliteralstrong{\sphinxupquote{repair\_order}} (\sphinxstyleliteralemphasis{\sphinxupquote{list of strings.}}) \textendash{} The repair order considered in the current simulation.

\end{description}\end{quote}

\end{fulllineitems}

\index{update\_directly\_affected\_components() (NetworkRecovery method)@\spxentry{update\_directly\_affected\_components()}\spxextra{NetworkRecovery method}}

\begin{fulllineitems}
\phantomsection\label{\detokenize{apidoc:dreaminsg_integrated_model.src.network_recovery.NetworkRecovery.update_directly_affected_components}}\pysiglinewithargsret{\sphinxbfcode{\sphinxupquote{update\_directly\_affected\_components}}}{\emph{\DUrole{n}{time\_stamp}}, \emph{\DUrole{n}{next\_sim\_time}}}{}
\sphinxAtStartPar
Updates the operational performance of directly impacted infrastructure components by the external event.
\begin{quote}\begin{description}
\item[{Parameters}] \leavevmode\begin{itemize}
\item {} 
\sphinxAtStartPar
\sphinxstyleliteralstrong{\sphinxupquote{time\_stamp}} (\sphinxstyleliteralemphasis{\sphinxupquote{integer}}) \textendash{} Current time stamp in the event table in seconds.

\item {} 
\sphinxAtStartPar
\sphinxstyleliteralstrong{\sphinxupquote{next\_sim\_time}} (\sphinxstyleliteralemphasis{\sphinxupquote{integer}}) \textendash{} Next time stamp in the event table in seconds.

\end{itemize}

\end{description}\end{quote}

\end{fulllineitems}

\index{update\_traffic\_model() (NetworkRecovery method)@\spxentry{update\_traffic\_model()}\spxextra{NetworkRecovery method}}

\begin{fulllineitems}
\phantomsection\label{\detokenize{apidoc:dreaminsg_integrated_model.src.network_recovery.NetworkRecovery.update_traffic_model}}\pysiglinewithargsret{\sphinxbfcode{\sphinxupquote{update\_traffic\_model}}}{}{}
\sphinxAtStartPar
Updates the static traffic assignment model based on current network conditions.

\end{fulllineitems}


\end{fulllineitems}

\index{link\_close\_event() (in module dreaminsg\_integrated\_model.src.network\_recovery)@\spxentry{link\_close\_event()}\spxextra{in module dreaminsg\_integrated\_model.src.network\_recovery}}

\begin{fulllineitems}
\phantomsection\label{\detokenize{apidoc:dreaminsg_integrated_model.src.network_recovery.link_close_event}}\pysiglinewithargsret{\sphinxbfcode{\sphinxupquote{link\_close\_event}}}{\emph{\DUrole{n}{wn}}, \emph{\DUrole{n}{pipe\_name}}, \emph{\DUrole{n}{time\_stamp}}, \emph{\DUrole{n}{state}}}{}
\sphinxAtStartPar
Closes a pipe.
\begin{quote}\begin{description}
\item[{Parameters}] \leavevmode\begin{itemize}
\item {} 
\sphinxAtStartPar
\sphinxstyleliteralstrong{\sphinxupquote{wn}} (\sphinxstyleliteralemphasis{\sphinxupquote{wntr network object}}) \textendash{} Water network object.

\item {} 
\sphinxAtStartPar
\sphinxstyleliteralstrong{\sphinxupquote{pipe\_name}} (\sphinxstyleliteralemphasis{\sphinxupquote{string}}) \textendash{} Name of the pipe.

\item {} 
\sphinxAtStartPar
\sphinxstyleliteralstrong{\sphinxupquote{time\_stamp}} (\sphinxstyleliteralemphasis{\sphinxupquote{integer}}) \textendash{} Time stamp at which the pipe must be closed in seconds.

\item {} 
\sphinxAtStartPar
\sphinxstyleliteralstrong{\sphinxupquote{state}} (\sphinxstyleliteralemphasis{\sphinxupquote{string}}) \textendash{} The state of the object.

\end{itemize}

\item[{Returns}] \leavevmode
\sphinxAtStartPar
The modified wntr network object after pipe splits.

\item[{Return type}] \leavevmode
\sphinxAtStartPar
wntr network object

\end{description}\end{quote}

\end{fulllineitems}

\index{link\_open\_event() (in module dreaminsg\_integrated\_model.src.network\_recovery)@\spxentry{link\_open\_event()}\spxextra{in module dreaminsg\_integrated\_model.src.network\_recovery}}

\begin{fulllineitems}
\phantomsection\label{\detokenize{apidoc:dreaminsg_integrated_model.src.network_recovery.link_open_event}}\pysiglinewithargsret{\sphinxbfcode{\sphinxupquote{link\_open\_event}}}{\emph{\DUrole{n}{wn}}, \emph{\DUrole{n}{pipe\_name}}, \emph{\DUrole{n}{time\_stamp}}, \emph{\DUrole{n}{state}}}{}
\sphinxAtStartPar
Opens a pipe.
\begin{quote}\begin{description}
\item[{Parameters}] \leavevmode\begin{itemize}
\item {} 
\sphinxAtStartPar
\sphinxstyleliteralstrong{\sphinxupquote{wn}} (\sphinxstyleliteralemphasis{\sphinxupquote{wntr network object}}) \textendash{} Water network object.

\item {} 
\sphinxAtStartPar
\sphinxstyleliteralstrong{\sphinxupquote{pipe\_name}} (\sphinxstyleliteralemphasis{\sphinxupquote{string}}) \textendash{} Name of the pipe.

\item {} 
\sphinxAtStartPar
\sphinxstyleliteralstrong{\sphinxupquote{time\_stamp}} (\sphinxstyleliteralemphasis{\sphinxupquote{integer}}) \textendash{} Time stamp at which the pipe must be opened in seconds.

\item {} 
\sphinxAtStartPar
\sphinxstyleliteralstrong{\sphinxupquote{state}} (\sphinxstyleliteralemphasis{\sphinxupquote{string}}) \textendash{} The state of the object.

\end{itemize}

\item[{Returns}] \leavevmode
\sphinxAtStartPar
The modified wntr network object after pipe splits.

\item[{Return type}] \leavevmode
\sphinxAtStartPar
wntr network object

\end{description}\end{quote}

\end{fulllineitems}

\index{pipe\_leak\_node\_generator() (in module dreaminsg\_integrated\_model.src.network\_recovery)@\spxentry{pipe\_leak\_node\_generator()}\spxextra{in module dreaminsg\_integrated\_model.src.network\_recovery}}

\begin{fulllineitems}
\phantomsection\label{\detokenize{apidoc:dreaminsg_integrated_model.src.network_recovery.pipe_leak_node_generator}}\pysiglinewithargsret{\sphinxbfcode{\sphinxupquote{pipe\_leak\_node\_generator}}}{\emph{\DUrole{n}{network}}}{}
\sphinxAtStartPar
Splits the directly affected pipes to induce leak during simulations.
\begin{quote}\begin{description}
\item[{Parameters}] \leavevmode\begin{itemize}
\item {} 
\sphinxAtStartPar
\sphinxstyleliteralstrong{\sphinxupquote{wn}} (\sphinxstyleliteralemphasis{\sphinxupquote{wntr network object}}) \textendash{} Water network object.

\item {} 
\sphinxAtStartPar
\sphinxstyleliteralstrong{\sphinxupquote{disaster\_recovery\_object}} (\sphinxstyleliteralemphasis{\sphinxupquote{DisasterAndRecovery object}}) \textendash{} The object in which all disaster and repair related information are stored.

\end{itemize}

\item[{Returns}] \leavevmode
\sphinxAtStartPar
The modified wntr network object after pipe splits.

\item[{Return type}] \leavevmode
\sphinxAtStartPar
wntr network object

\end{description}\end{quote}

\end{fulllineitems}



\subsection{dreaminsg\_integrated\_model.src.network\_generator}
\label{\detokenize{apidoc:dreaminsg-integrated-model-src-network-generator}}\phantomsection\label{\detokenize{apidoc:module-dreaminsg_integrated_model.src.network_generator}}\index{module@\spxentry{module}!dreaminsg\_integrated\_model.src.network\_generator@\spxentry{dreaminsg\_integrated\_model.src.network\_generator}}\index{dreaminsg\_integrated\_model.src.network\_generator@\spxentry{dreaminsg\_integrated\_model.src.network\_generator}!module@\spxentry{module}}
\sphinxAtStartPar
Functions to generate the infrastructure networks used for simulation.
\index{generate\_powern() (in module dreaminsg\_integrated\_model.src.network\_generator)@\spxentry{generate\_powern()}\spxextra{in module dreaminsg\_integrated\_model.src.network\_generator}}

\begin{fulllineitems}
\phantomsection\label{\detokenize{apidoc:dreaminsg_integrated_model.src.network_generator.generate_powern}}\pysiglinewithargsret{\sphinxbfcode{\sphinxupquote{generate\_powern}}}{\emph{\DUrole{n}{file\_name}}}{}
\sphinxAtStartPar
Generates a power system network using the pandapower package and saves it to local directory.
\begin{quote}\begin{description}
\item[{Parameters}] \leavevmode
\sphinxAtStartPar
\sphinxstyleliteralstrong{\sphinxupquote{file\_name}} (\sphinxstyleliteralemphasis{\sphinxupquote{string}}) \textendash{} Name of the {\color{red}\bfseries{}*}.json file to be saved including path.

\end{description}\end{quote}

\end{fulllineitems}

\index{generate\_watern() (in module dreaminsg\_integrated\_model.src.network\_generator)@\spxentry{generate\_watern()}\spxextra{in module dreaminsg\_integrated\_model.src.network\_generator}}

\begin{fulllineitems}
\phantomsection\label{\detokenize{apidoc:dreaminsg_integrated_model.src.network_generator.generate_watern}}\pysiglinewithargsret{\sphinxbfcode{\sphinxupquote{generate\_watern}}}{\emph{\DUrole{n}{file\_name}}}{}
\sphinxAtStartPar
Generates a water network using the wntr package and saves it to local directory.
\begin{quote}\begin{description}
\item[{Parameters}] \leavevmode
\sphinxAtStartPar
\sphinxstyleliteralstrong{\sphinxupquote{file\_name}} (\sphinxstyleliteralemphasis{\sphinxupquote{string}}) \textendash{} Name of the {\color{red}\bfseries{}*}.inp file to be saved including path.

\end{description}\end{quote}

\end{fulllineitems}



\subsection{dreaminsg\_integrated\_model.src.simulation}
\label{\detokenize{apidoc:module-dreaminsg_integrated_model.src.simulation}}\label{\detokenize{apidoc:dreaminsg-integrated-model-src-simulation}}\index{module@\spxentry{module}!dreaminsg\_integrated\_model.src.simulation@\spxentry{dreaminsg\_integrated\_model.src.simulation}}\index{dreaminsg\_integrated\_model.src.simulation@\spxentry{dreaminsg\_integrated\_model.src.simulation}!module@\spxentry{module}}
\sphinxAtStartPar
Functions to implement the various steps of the interdependent infrastructure network simulations.
\index{NetworkSimulation (class in dreaminsg\_integrated\_model.src.simulation)@\spxentry{NetworkSimulation}\spxextra{class in dreaminsg\_integrated\_model.src.simulation}}

\begin{fulllineitems}
\phantomsection\label{\detokenize{apidoc:dreaminsg_integrated_model.src.simulation.NetworkSimulation}}\pysiglinewithargsret{\sphinxbfcode{\sphinxupquote{class }}\sphinxbfcode{\sphinxupquote{NetworkSimulation}}}{\emph{\DUrole{n}{network\_recovery}}, \emph{\DUrole{n}{sim\_step}}}{}
\sphinxAtStartPar
Bases: \sphinxcode{\sphinxupquote{object}}

\sphinxAtStartPar
Methods to perform simulation of interdependent effects.
\index{expand\_event\_table() (NetworkSimulation method)@\spxentry{expand\_event\_table()}\spxextra{NetworkSimulation method}}

\begin{fulllineitems}
\phantomsection\label{\detokenize{apidoc:dreaminsg_integrated_model.src.simulation.NetworkSimulation.expand_event_table}}\pysiglinewithargsret{\sphinxbfcode{\sphinxupquote{expand\_event\_table}}}{\emph{\DUrole{n}{add\_points}}}{}
\sphinxAtStartPar
Expands the event table with additional time\_stamps for simulation.
\begin{quote}\begin{description}
\item[{Parameters}] \leavevmode
\sphinxAtStartPar
\sphinxstyleliteralstrong{\sphinxupquote{add\_points}} (\sphinxstyleliteralemphasis{\sphinxupquote{integer}}) \textendash{} A positive integer denoting the number of extra time\sphinxhyphen{}stamps to be added to the simulation.

\end{description}\end{quote}

\end{fulllineitems}

\index{get\_components\_repaired() (NetworkSimulation method)@\spxentry{get\_components\_repaired()}\spxextra{NetworkSimulation method}}

\begin{fulllineitems}
\phantomsection\label{\detokenize{apidoc:dreaminsg_integrated_model.src.simulation.NetworkSimulation.get_components_repaired}}\pysiglinewithargsret{\sphinxbfcode{\sphinxupquote{get\_components\_repaired}}}{}{}
\sphinxAtStartPar
Returns the list of components that are already repaired.
\begin{quote}\begin{description}
\item[{Returns}] \leavevmode
\sphinxAtStartPar
list of components which are already repaired.

\item[{Return type}] \leavevmode
\sphinxAtStartPar
list of strings

\end{description}\end{quote}

\end{fulllineitems}

\index{get\_components\_to\_repair() (NetworkSimulation method)@\spxentry{get\_components\_to\_repair()}\spxextra{NetworkSimulation method}}

\begin{fulllineitems}
\phantomsection\label{\detokenize{apidoc:dreaminsg_integrated_model.src.simulation.NetworkSimulation.get_components_to_repair}}\pysiglinewithargsret{\sphinxbfcode{\sphinxupquote{get\_components\_to\_repair}}}{}{}
\sphinxAtStartPar
Returns the remaining components to be repaired.
\begin{quote}\begin{description}
\item[{Returns}] \leavevmode
\sphinxAtStartPar
The list of components

\item[{Return type}] \leavevmode
\sphinxAtStartPar
list of strings

\end{description}\end{quote}

\end{fulllineitems}

\index{simulate\_interdependent\_effects() (NetworkSimulation method)@\spxentry{simulate\_interdependent\_effects()}\spxextra{NetworkSimulation method}}

\begin{fulllineitems}
\phantomsection\label{\detokenize{apidoc:dreaminsg_integrated_model.src.simulation.NetworkSimulation.simulate_interdependent_effects}}\pysiglinewithargsret{\sphinxbfcode{\sphinxupquote{simulate\_interdependent\_effects}}}{\emph{\DUrole{n}{network\_recovery}}}{}
\sphinxAtStartPar
Simulates the interdependent effect based on the initial disruptions and subsequent repair actions.
\begin{quote}\begin{description}
\item[{Parameters}] \leavevmode
\sphinxAtStartPar
\sphinxstyleliteralstrong{\sphinxupquote{network\_recovery}} (\sphinxstyleliteralemphasis{\sphinxupquote{NetworkRecovery object}}) \textendash{} A integrated infrastructure network recovery object.

\item[{Returns}] \leavevmode
\sphinxAtStartPar
lists of time stamps and resilience values of power and water supply.

\item[{Return type}] \leavevmode
\sphinxAtStartPar
lists

\end{description}\end{quote}

\end{fulllineitems}

\index{update\_repaired\_components() (NetworkSimulation method)@\spxentry{update\_repaired\_components()}\spxextra{NetworkSimulation method}}

\begin{fulllineitems}
\phantomsection\label{\detokenize{apidoc:dreaminsg_integrated_model.src.simulation.NetworkSimulation.update_repaired_components}}\pysiglinewithargsret{\sphinxbfcode{\sphinxupquote{update\_repaired\_components}}}{\emph{\DUrole{n}{component}}}{}
\sphinxAtStartPar
Update the lists of repaired and to be repaired components.
\begin{quote}\begin{description}
\item[{Parameters}] \leavevmode
\sphinxAtStartPar
\sphinxstyleliteralstrong{\sphinxupquote{component}} (\sphinxstyleliteralemphasis{\sphinxupquote{string}}) \textendash{} The name of the component that was recently repaired.

\end{description}\end{quote}

\end{fulllineitems}

\index{write\_results() (NetworkSimulation method)@\spxentry{write\_results()}\spxextra{NetworkSimulation method}}

\begin{fulllineitems}
\phantomsection\label{\detokenize{apidoc:dreaminsg_integrated_model.src.simulation.NetworkSimulation.write_results}}\pysiglinewithargsret{\sphinxbfcode{\sphinxupquote{write\_results}}}{\emph{\DUrole{n}{time\_tracker}}, \emph{\DUrole{n}{power\_consump\_tracker}}, \emph{\DUrole{n}{water\_consump\_tracker}}, \emph{\DUrole{n}{location}}, \emph{\DUrole{n}{plotting}\DUrole{o}{=}\DUrole{default_value}{False}}}{}
\sphinxAtStartPar
Write the results to local directory.
\begin{quote}\begin{description}
\item[{Parameters}] \leavevmode\begin{itemize}
\item {} 
\sphinxAtStartPar
\sphinxstyleliteralstrong{\sphinxupquote{time\_tracker}} (\sphinxstyleliteralemphasis{\sphinxupquote{list of integers}}) \textendash{} List of time stamps.

\item {} 
\sphinxAtStartPar
\sphinxstyleliteralstrong{\sphinxupquote{power\_consump\_tracker}} (\sphinxstyleliteralemphasis{\sphinxupquote{list of floats}}) \textendash{} List of corresponding power resilience metric value.

\item {} 
\sphinxAtStartPar
\sphinxstyleliteralstrong{\sphinxupquote{water\_consump\_tracker}} (\sphinxstyleliteralemphasis{\sphinxupquote{list of floats}}) \textendash{} List of corresponding water resilience metric value.

\item {} 
\sphinxAtStartPar
\sphinxstyleliteralstrong{\sphinxupquote{location}} (\sphinxstyleliteralemphasis{\sphinxupquote{string}}) \textendash{} The location to which the results are to be saved.

\item {} 
\sphinxAtStartPar
\sphinxstyleliteralstrong{\sphinxupquote{plotting}} (\sphinxstyleliteralemphasis{\sphinxupquote{bool}}\sphinxstyleliteralemphasis{\sphinxupquote{, }}\sphinxstyleliteralemphasis{\sphinxupquote{optional}}) \textendash{} True if the plots are to be generated., defaults to False

\end{itemize}

\end{description}\end{quote}

\end{fulllineitems}


\end{fulllineitems}



\subsection{dreaminsg\_integrated\_model.src.optimizer}
\label{\detokenize{apidoc:module-dreaminsg_integrated_model.src.optimizer}}\label{\detokenize{apidoc:dreaminsg-integrated-model-src-optimizer}}\index{module@\spxentry{module}!dreaminsg\_integrated\_model.src.optimizer@\spxentry{dreaminsg\_integrated\_model.src.optimizer}}\index{dreaminsg\_integrated\_model.src.optimizer@\spxentry{dreaminsg\_integrated\_model.src.optimizer}!module@\spxentry{module}}\index{BruteForceOptimizer (class in dreaminsg\_integrated\_model.src.optimizer)@\spxentry{BruteForceOptimizer}\spxextra{class in dreaminsg\_integrated\_model.src.optimizer}}

\begin{fulllineitems}
\phantomsection\label{\detokenize{apidoc:dreaminsg_integrated_model.src.optimizer.BruteForceOptimizer}}\pysiglinewithargsret{\sphinxbfcode{\sphinxupquote{class }}\sphinxbfcode{\sphinxupquote{BruteForceOptimizer}}}{\emph{\DUrole{n}{prediction\_horizon}\DUrole{o}{=}\DUrole{default_value}{None}}}{}
\sphinxAtStartPar
Bases: {\hyperref[\detokenize{apidoc:dreaminsg_integrated_model.src.optimizer.Optimizer}]{\sphinxcrossref{\sphinxcode{\sphinxupquote{dreaminsg\_integrated\_model.src.optimizer.Optimizer}}}}}

\sphinxAtStartPar
A Brute Force Optimizer class
\begin{quote}\begin{description}
\item[{Parameters}] \leavevmode
\sphinxAtStartPar
\sphinxstyleliteralstrong{\sphinxupquote{Optimizer}} (\sphinxstyleliteralemphasis{\sphinxupquote{Optimizer abstract class.}}) \textendash{} An optimizer class.

\end{description}\end{quote}
\index{find\_optimal\_recovery() (BruteForceOptimizer method)@\spxentry{find\_optimal\_recovery()}\spxextra{BruteForceOptimizer method}}

\begin{fulllineitems}
\phantomsection\label{\detokenize{apidoc:dreaminsg_integrated_model.src.optimizer.BruteForceOptimizer.find_optimal_recovery}}\pysiglinewithargsret{\sphinxbfcode{\sphinxupquote{find\_optimal\_recovery}}}{\emph{\DUrole{n}{simulation}}}{}
\sphinxAtStartPar
Identifies the optimal recovery strategy using the Model Predictive Control principle.
\begin{quote}\begin{description}
\item[{Parameters}] \leavevmode
\sphinxAtStartPar
\sphinxstyleliteralstrong{\sphinxupquote{simulation}} (\sphinxstyleliteralemphasis{\sphinxupquote{Simulation object.}}) \textendash{} The infrastructure network simulation object.

\end{description}\end{quote}

\end{fulllineitems}

\index{get\_optimization\_log() (BruteForceOptimizer method)@\spxentry{get\_optimization\_log()}\spxextra{BruteForceOptimizer method}}

\begin{fulllineitems}
\phantomsection\label{\detokenize{apidoc:dreaminsg_integrated_model.src.optimizer.BruteForceOptimizer.get_optimization_log}}\pysiglinewithargsret{\sphinxbfcode{\sphinxupquote{get\_optimization\_log}}}{}{}
\sphinxAtStartPar
Returns the optimization log.
\begin{quote}\begin{description}
\item[{Returns}] \leavevmode
\sphinxAtStartPar
A table consisting of the AUC values from the network simulations.

\item[{Return type}] \leavevmode
\sphinxAtStartPar
pandas dataframe.

\end{description}\end{quote}

\end{fulllineitems}

\index{get\_repair\_permutations() (BruteForceOptimizer method)@\spxentry{get\_repair\_permutations()}\spxextra{BruteForceOptimizer method}}

\begin{fulllineitems}
\phantomsection\label{\detokenize{apidoc:dreaminsg_integrated_model.src.optimizer.BruteForceOptimizer.get_repair_permutations}}\pysiglinewithargsret{\sphinxbfcode{\sphinxupquote{get\_repair\_permutations}}}{\emph{\DUrole{n}{simulation}}}{}
\sphinxAtStartPar
Returns all possible permutations of the repair order.
\begin{quote}\begin{description}
\item[{Parameters}] \leavevmode
\sphinxAtStartPar
\sphinxstyleliteralstrong{\sphinxupquote{simulation}} (\sphinxstyleliteralemphasis{\sphinxupquote{Simulation object}}) \textendash{} An integrated infrastructure network simulation object.

\item[{Returns}] \leavevmode
\sphinxAtStartPar
A nested list of all possible repair permutations for the given list of components.

\item[{Return type}] \leavevmode
\sphinxAtStartPar
list of lists of strings.

\end{description}\end{quote}

\end{fulllineitems}

\index{get\_trackers() (BruteForceOptimizer method)@\spxentry{get\_trackers()}\spxextra{BruteForceOptimizer method}}

\begin{fulllineitems}
\phantomsection\label{\detokenize{apidoc:dreaminsg_integrated_model.src.optimizer.BruteForceOptimizer.get_trackers}}\pysiglinewithargsret{\sphinxbfcode{\sphinxupquote{get\_trackers}}}{}{}
\sphinxAtStartPar
Returns the time, power consumption ratio and water consumption ratio values.
\begin{description}
\item[{Returns:}] \leavevmode
\sphinxAtStartPar
lists: lists of lists

\end{description}

\end{fulllineitems}


\end{fulllineitems}

\index{Optimizer (class in dreaminsg\_integrated\_model.src.optimizer)@\spxentry{Optimizer}\spxextra{class in dreaminsg\_integrated\_model.src.optimizer}}

\begin{fulllineitems}
\phantomsection\label{\detokenize{apidoc:dreaminsg_integrated_model.src.optimizer.Optimizer}}\pysiglinewithargsret{\sphinxbfcode{\sphinxupquote{class }}\sphinxbfcode{\sphinxupquote{Optimizer}}}{\emph{\DUrole{n}{prediction\_horizon}\DUrole{o}{=}\DUrole{default_value}{None}}}{}
\sphinxAtStartPar
Bases: \sphinxcode{\sphinxupquote{abc.ABC}}

\sphinxAtStartPar
The Optimizer class defines an interface to a discrete optimizer or can be implemented as such. This optimizer takes a network object and a prediction horizon and should compute the best steps of the length of the prediction\_horizon

\end{fulllineitems}



\subsection{dreaminsg\_integrated\_model.src.resilience\_metrics}
\label{\detokenize{apidoc:dreaminsg-integrated-model-src-resilience-metrics}}\phantomsection\label{\detokenize{apidoc:module-dreaminsg_integrated_model.src.resilience_metrics}}\index{module@\spxentry{module}!dreaminsg\_integrated\_model.src.resilience\_metrics@\spxentry{dreaminsg\_integrated\_model.src.resilience\_metrics}}\index{dreaminsg\_integrated\_model.src.resilience\_metrics@\spxentry{dreaminsg\_integrated\_model.src.resilience\_metrics}!module@\spxentry{module}}
\sphinxAtStartPar
Resilience metric classes to be used for optimizing recovery actions.
\index{ResilienceMetric (class in dreaminsg\_integrated\_model.src.resilience\_metrics)@\spxentry{ResilienceMetric}\spxextra{class in dreaminsg\_integrated\_model.src.resilience\_metrics}}

\begin{fulllineitems}
\phantomsection\label{\detokenize{apidoc:dreaminsg_integrated_model.src.resilience_metrics.ResilienceMetric}}\pysigline{\sphinxbfcode{\sphinxupquote{class }}\sphinxbfcode{\sphinxupquote{ResilienceMetric}}}
\sphinxAtStartPar
Bases: \sphinxcode{\sphinxupquote{abc.ABC}}

\sphinxAtStartPar
The ResilienceMetric class defines an interface to a resilience metric.

\end{fulllineitems}

\index{WeightedResilienceMetric (class in dreaminsg\_integrated\_model.src.resilience\_metrics)@\spxentry{WeightedResilienceMetric}\spxextra{class in dreaminsg\_integrated\_model.src.resilience\_metrics}}

\begin{fulllineitems}
\phantomsection\label{\detokenize{apidoc:dreaminsg_integrated_model.src.resilience_metrics.WeightedResilienceMetric}}\pysigline{\sphinxbfcode{\sphinxupquote{class }}\sphinxbfcode{\sphinxupquote{WeightedResilienceMetric}}}
\sphinxAtStartPar
Bases: \sphinxcode{\sphinxupquote{dreaminsg\_integrated\_model.src.resilience\_metrics.ResilienceMetric}}

\sphinxAtStartPar
Methods to calculated the weighted ILOS estimates without normalization.
\index{calculate\_power\_resmetric() (WeightedResilienceMetric method)@\spxentry{calculate\_power\_resmetric()}\spxextra{WeightedResilienceMetric method}}

\begin{fulllineitems}
\phantomsection\label{\detokenize{apidoc:dreaminsg_integrated_model.src.resilience_metrics.WeightedResilienceMetric.calculate_power_resmetric}}\pysiglinewithargsret{\sphinxbfcode{\sphinxupquote{calculate\_power\_resmetric}}}{\emph{\DUrole{n}{network\_recovery}}}{}
\sphinxAtStartPar
Calcualtes the power resilience metric.
\begin{quote}\begin{description}
\item[{Parameters}] \leavevmode
\sphinxAtStartPar
\sphinxstyleliteralstrong{\sphinxupquote{network\_recovery}} (\sphinxstyleliteralemphasis{\sphinxupquote{NetworkRecovery object}}) \textendash{} The network recovery object

\item[{Returns}] \leavevmode
\sphinxAtStartPar
Power resilience metric value

\item[{Return type}] \leavevmode
\sphinxAtStartPar
float

\end{description}\end{quote}

\end{fulllineitems}

\index{calculate\_water\_resmetric() (WeightedResilienceMetric method)@\spxentry{calculate\_water\_resmetric()}\spxextra{WeightedResilienceMetric method}}

\begin{fulllineitems}
\phantomsection\label{\detokenize{apidoc:dreaminsg_integrated_model.src.resilience_metrics.WeightedResilienceMetric.calculate_water_resmetric}}\pysiglinewithargsret{\sphinxbfcode{\sphinxupquote{calculate\_water\_resmetric}}}{\emph{\DUrole{n}{network\_recovery}}, \emph{\DUrole{n}{wn\_results}}}{}
\sphinxAtStartPar
Calculates and returns the water resilience metric.
\begin{quote}\begin{description}
\item[{Parameters}] \leavevmode\begin{itemize}
\item {} 
\sphinxAtStartPar
\sphinxstyleliteralstrong{\sphinxupquote{network\_recovery}} (\sphinxstyleliteralemphasis{\sphinxupquote{NetworkRecovery object}}) \textendash{} The network recovery object

\item {} 
\sphinxAtStartPar
\sphinxstyleliteralstrong{\sphinxupquote{wn\_results}} (\sphinxstyleliteralemphasis{\sphinxupquote{wntr object}}) \textendash{} The water network simulation results for the current time interval

\end{itemize}

\item[{Returns}] \leavevmode
\sphinxAtStartPar
water resilience metric value

\item[{Return type}] \leavevmode
\sphinxAtStartPar
float

\end{description}\end{quote}

\end{fulllineitems}


\end{fulllineitems}



\subsection{dreaminsg\_integrated\_model.src.plots}
\label{\detokenize{apidoc:module-dreaminsg_integrated_model.src.plots}}\label{\detokenize{apidoc:dreaminsg-integrated-model-src-plots}}\index{module@\spxentry{module}!dreaminsg\_integrated\_model.src.plots@\spxentry{dreaminsg\_integrated\_model.src.plots}}\index{dreaminsg\_integrated\_model.src.plots@\spxentry{dreaminsg\_integrated\_model.src.plots}!module@\spxentry{module}}
\sphinxAtStartPar
Functions to generate infrastructure network plots and result plots.
\index{plot\_integrated\_network() (in module dreaminsg\_integrated\_model.src.plots)@\spxentry{plot\_integrated\_network()}\spxextra{in module dreaminsg\_integrated\_model.src.plots}}

\begin{fulllineitems}
\phantomsection\label{\detokenize{apidoc:dreaminsg_integrated_model.src.plots.plot_integrated_network}}\pysiglinewithargsret{\sphinxbfcode{\sphinxupquote{plot\_integrated\_network}}}{\emph{\DUrole{n}{pn}}, \emph{\DUrole{n}{wn}}, \emph{\DUrole{n}{tn}}, \emph{\DUrole{n}{plotting}\DUrole{o}{=}\DUrole{default_value}{False}}}{}
\sphinxAtStartPar
Generates the integrated networkx object.
\begin{quote}\begin{description}
\item[{Parameters}] \leavevmode\begin{itemize}
\item {} 
\sphinxAtStartPar
\sphinxstyleliteralstrong{\sphinxupquote{pn}} (\sphinxstyleliteralemphasis{\sphinxupquote{pandapower network object}}) \textendash{} Power network object.

\item {} 
\sphinxAtStartPar
\sphinxstyleliteralstrong{\sphinxupquote{wn}} (\sphinxstyleliteralemphasis{\sphinxupquote{wntr network object}}) \textendash{} Water network object.

\item {} 
\sphinxAtStartPar
\sphinxstyleliteralstrong{\sphinxupquote{tn}} (\sphinxstyleliteralemphasis{\sphinxupquote{STA network object}}) \textendash{} Traffic network object.

\item {} 
\sphinxAtStartPar
\sphinxstyleliteralstrong{\sphinxupquote{plotting}} (\sphinxstyleliteralemphasis{\sphinxupquote{bool}}\sphinxstyleliteralemphasis{\sphinxupquote{, }}\sphinxstyleliteralemphasis{\sphinxupquote{optional}}) \textendash{} Generates plots, defaults to False.

\end{itemize}

\item[{Returns}] \leavevmode
\sphinxAtStartPar
The integrated infrastructure graph.

\item[{Return type}] \leavevmode
\sphinxAtStartPar
networkx object

\end{description}\end{quote}

\end{fulllineitems}

\index{plot\_interdependent\_effects() (in module dreaminsg\_integrated\_model.src.plots)@\spxentry{plot\_interdependent\_effects()}\spxextra{in module dreaminsg\_integrated\_model.src.plots}}

\begin{fulllineitems}
\phantomsection\label{\detokenize{apidoc:dreaminsg_integrated_model.src.plots.plot_interdependent_effects}}\pysiglinewithargsret{\sphinxbfcode{\sphinxupquote{plot\_interdependent\_effects}}}{\emph{\DUrole{n}{power\_consump\_tracker}}, \emph{\DUrole{n}{water\_consump\_tracker}}, \emph{\DUrole{n}{time\_tracker}}, \emph{\DUrole{n}{scatter}\DUrole{o}{=}\DUrole{default_value}{True}}}{}
\sphinxAtStartPar
Generates the network\sphinxhyphen{}level performance plots.
\begin{quote}\begin{description}
\item[{Parameters}] \leavevmode\begin{itemize}
\item {} 
\sphinxAtStartPar
\sphinxstyleliteralstrong{\sphinxupquote{power\_consump\_tracker}} (\sphinxstyleliteralemphasis{\sphinxupquote{list of floats}}) \textendash{} A list of power consumption resilience metric values.

\item {} 
\sphinxAtStartPar
\sphinxstyleliteralstrong{\sphinxupquote{water\_consump\_tracker}} (\sphinxstyleliteralemphasis{\sphinxupquote{list of floats}}) \textendash{} A list of water consumption resilience metric values.

\item {} 
\sphinxAtStartPar
\sphinxstyleliteralstrong{\sphinxupquote{time\_tracker}} (\sphinxstyleliteralemphasis{\sphinxupquote{list of floats}}) \textendash{} A list of time\sphinxhyphen{}stamps from the similation.

\item {} 
\sphinxAtStartPar
\sphinxstyleliteralstrong{\sphinxupquote{scatter}} (\sphinxstyleliteralemphasis{\sphinxupquote{bool}}\sphinxstyleliteralemphasis{\sphinxupquote{, }}\sphinxstyleliteralemphasis{\sphinxupquote{optional}}) \textendash{} scatter plot, defaults to True

\end{itemize}

\end{description}\end{quote}

\end{fulllineitems}

\index{plot\_power\_net() (in module dreaminsg\_integrated\_model.src.plots)@\spxentry{plot\_power\_net()}\spxextra{in module dreaminsg\_integrated\_model.src.plots}}

\begin{fulllineitems}
\phantomsection\label{\detokenize{apidoc:dreaminsg_integrated_model.src.plots.plot_power_net}}\pysiglinewithargsret{\sphinxbfcode{\sphinxupquote{plot\_power\_net}}}{\emph{\DUrole{n}{net}}}{}
\sphinxAtStartPar
Generates the power systems plot.
\begin{quote}\begin{description}
\item[{Parameters}] \leavevmode
\sphinxAtStartPar
\sphinxstyleliteralstrong{\sphinxupquote{net}} (\sphinxstyleliteralemphasis{\sphinxupquote{pandapower network object.}}) \textendash{} The power systems network.

\end{description}\end{quote}

\end{fulllineitems}

\index{plot\_repair\_curves() (in module dreaminsg\_integrated\_model.src.plots)@\spxentry{plot\_repair\_curves()}\spxextra{in module dreaminsg\_integrated\_model.src.plots}}

\begin{fulllineitems}
\phantomsection\label{\detokenize{apidoc:dreaminsg_integrated_model.src.plots.plot_repair_curves}}\pysiglinewithargsret{\sphinxbfcode{\sphinxupquote{plot\_repair\_curves}}}{\emph{\DUrole{n}{disrupt\_recovery\_object}}, \emph{\DUrole{n}{scatter}\DUrole{o}{=}\DUrole{default_value}{False}}}{}
\sphinxAtStartPar
Generates the direct impact and repair level plots for the failed components.
\begin{quote}\begin{description}
\item[{Parameters}] \leavevmode\begin{itemize}
\item {} 
\sphinxAtStartPar
\sphinxstyleliteralstrong{\sphinxupquote{disrupt\_recovery\_object}} (\sphinxstyleliteralemphasis{\sphinxupquote{DisasterAndRecovery object}}) \textendash{} The disrupt\_generator.DisruptionAndRecovery object.

\item {} 
\sphinxAtStartPar
\sphinxstyleliteralstrong{\sphinxupquote{scatter}} (\sphinxstyleliteralemphasis{\sphinxupquote{bool}}\sphinxstyleliteralemphasis{\sphinxupquote{, }}\sphinxstyleliteralemphasis{\sphinxupquote{optional}}) \textendash{} scatter plot, defaults to False

\end{itemize}

\end{description}\end{quote}

\end{fulllineitems}

\index{plot\_transpo\_net() (in module dreaminsg\_integrated\_model.src.plots)@\spxentry{plot\_transpo\_net()}\spxextra{in module dreaminsg\_integrated\_model.src.plots}}

\begin{fulllineitems}
\phantomsection\label{\detokenize{apidoc:dreaminsg_integrated_model.src.plots.plot_transpo_net}}\pysiglinewithargsret{\sphinxbfcode{\sphinxupquote{plot\_transpo\_net}}}{\emph{\DUrole{n}{transpo\_folder}}}{}
\sphinxAtStartPar
Generates the transportation network plot.
\begin{quote}\begin{description}
\item[{Parameters}] \leavevmode
\sphinxAtStartPar
\sphinxstyleliteralstrong{\sphinxupquote{transpo\_folder}} (\sphinxstyleliteralemphasis{\sphinxupquote{string}}) \textendash{} Location of the .tntp files.

\end{description}\end{quote}

\end{fulllineitems}

\index{plot\_water\_net() (in module dreaminsg\_integrated\_model.src.plots)@\spxentry{plot\_water\_net()}\spxextra{in module dreaminsg\_integrated\_model.src.plots}}

\begin{fulllineitems}
\phantomsection\label{\detokenize{apidoc:dreaminsg_integrated_model.src.plots.plot_water_net}}\pysiglinewithargsret{\sphinxbfcode{\sphinxupquote{plot\_water\_net}}}{\emph{\DUrole{n}{wn}}}{}
\sphinxAtStartPar
Generates the water network plot.
\begin{quote}\begin{description}
\item[{Parameters}] \leavevmode
\sphinxAtStartPar
\sphinxstyleliteralstrong{\sphinxupquote{wn}} (\sphinxstyleliteralemphasis{\sphinxupquote{wntr network object.}}) \textendash{} The water network.

\end{description}\end{quote}

\end{fulllineitems}



\subsection{dreaminsg\_integrated\_model.src.network\_sim\_models.integrated\_network}
\label{\detokenize{apidoc:dreaminsg-integrated-model-src-network-sim-models-integrated-network}}\phantomsection\label{\detokenize{apidoc:module-dreaminsg_integrated_model.src.network_sim_models.integrated_network}}\index{module@\spxentry{module}!dreaminsg\_integrated\_model.src.network\_sim\_models.integrated\_network@\spxentry{dreaminsg\_integrated\_model.src.network\_sim\_models.integrated\_network}}\index{dreaminsg\_integrated\_model.src.network\_sim\_models.integrated\_network@\spxentry{dreaminsg\_integrated\_model.src.network\_sim\_models.integrated\_network}!module@\spxentry{module}}\index{IntegratedNetwork (class in dreaminsg\_integrated\_model.src.network\_sim\_models.integrated\_network)@\spxentry{IntegratedNetwork}\spxextra{class in dreaminsg\_integrated\_model.src.network\_sim\_models.integrated\_network}}

\begin{fulllineitems}
\phantomsection\label{\detokenize{apidoc:dreaminsg_integrated_model.src.network_sim_models.integrated_network.IntegratedNetwork}}\pysigline{\sphinxbfcode{\sphinxupquote{class }}\sphinxbfcode{\sphinxupquote{IntegratedNetwork}}}
\sphinxAtStartPar
Bases: {\hyperref[\detokenize{apidoc:dreaminsg_integrated_model.src.network_sim_models.integrated_network.Network}]{\sphinxcrossref{\sphinxcode{\sphinxupquote{dreaminsg\_integrated\_model.src.network\_sim\_models.integrated\_network.Network}}}}}

\sphinxAtStartPar
An integrated infrastructure network class
\index{generate\_dependency\_table() (IntegratedNetwork method)@\spxentry{generate\_dependency\_table()}\spxextra{IntegratedNetwork method}}

\begin{fulllineitems}
\phantomsection\label{\detokenize{apidoc:dreaminsg_integrated_model.src.network_sim_models.integrated_network.IntegratedNetwork.generate_dependency_table}}\pysiglinewithargsret{\sphinxbfcode{\sphinxupquote{generate\_dependency\_table}}}{\emph{\DUrole{n}{dependency\_file}}}{}
\sphinxAtStartPar
Generates the dependency table from an input file.
\begin{quote}\begin{description}
\item[{Parameters}] \leavevmode
\sphinxAtStartPar
\sphinxstyleliteralstrong{\sphinxupquote{dependency\_file}} (\sphinxstyleliteralemphasis{\sphinxupquote{string}}) \textendash{} The location of the dependency file in csv format.

\end{description}\end{quote}

\end{fulllineitems}

\index{generate\_integrated\_graph() (IntegratedNetwork method)@\spxentry{generate\_integrated\_graph()}\spxextra{IntegratedNetwork method}}

\begin{fulllineitems}
\phantomsection\label{\detokenize{apidoc:dreaminsg_integrated_model.src.network_sim_models.integrated_network.IntegratedNetwork.generate_integrated_graph}}\pysiglinewithargsret{\sphinxbfcode{\sphinxupquote{generate\_integrated\_graph}}}{\emph{\DUrole{n}{plotting}\DUrole{o}{=}\DUrole{default_value}{False}}, \emph{\DUrole{n}{legend\_size}\DUrole{o}{=}\DUrole{default_value}{12}}, \emph{\DUrole{n}{font\_size}\DUrole{o}{=}\DUrole{default_value}{8}}, \emph{\DUrole{n}{figsize}\DUrole{o}{=}\DUrole{default_value}{(10, 7)}}, \emph{\DUrole{n}{line\_width}\DUrole{o}{=}\DUrole{default_value}{2}}}{}
\sphinxAtStartPar
Generates the integrated Networkx object.
\begin{quote}\begin{description}
\item[{Parameters}] \leavevmode\begin{itemize}
\item {} 
\sphinxAtStartPar
\sphinxstyleliteralstrong{\sphinxupquote{plotting}} (\sphinxstyleliteralemphasis{\sphinxupquote{bool}}\sphinxstyleliteralemphasis{\sphinxupquote{, }}\sphinxstyleliteralemphasis{\sphinxupquote{optional}}) \textendash{} Generates plots, defaults to False., defaults to False

\item {} 
\sphinxAtStartPar
\sphinxstyleliteralstrong{\sphinxupquote{legend\_size}} (\sphinxstyleliteralemphasis{\sphinxupquote{int}}\sphinxstyleliteralemphasis{\sphinxupquote{, }}\sphinxstyleliteralemphasis{\sphinxupquote{optional}}) \textendash{} Legend font size, defaults to 12

\item {} 
\sphinxAtStartPar
\sphinxstyleliteralstrong{\sphinxupquote{font\_size}} (\sphinxstyleliteralemphasis{\sphinxupquote{int}}\sphinxstyleliteralemphasis{\sphinxupquote{, }}\sphinxstyleliteralemphasis{\sphinxupquote{optional}}) \textendash{} Text font size, defaults to 8

\item {} 
\sphinxAtStartPar
\sphinxstyleliteralstrong{\sphinxupquote{figsize}} (\sphinxstyleliteralemphasis{\sphinxupquote{tuple}}\sphinxstyleliteralemphasis{\sphinxupquote{, }}\sphinxstyleliteralemphasis{\sphinxupquote{optional}}) \textendash{} Size of final figure, defaults to (10, 7)

\item {} 
\sphinxAtStartPar
\sphinxstyleliteralstrong{\sphinxupquote{line\_width}} (\sphinxstyleliteralemphasis{\sphinxupquote{int}}\sphinxstyleliteralemphasis{\sphinxupquote{, }}\sphinxstyleliteralemphasis{\sphinxupquote{optional}}) \textendash{} Width of lines, defaults to 2

\end{itemize}

\end{description}\end{quote}

\end{fulllineitems}

\index{get\_disrupted\_components() (IntegratedNetwork method)@\spxentry{get\_disrupted\_components()}\spxextra{IntegratedNetwork method}}

\begin{fulllineitems}
\phantomsection\label{\detokenize{apidoc:dreaminsg_integrated_model.src.network_sim_models.integrated_network.IntegratedNetwork.get_disrupted_components}}\pysiglinewithargsret{\sphinxbfcode{\sphinxupquote{get\_disrupted\_components}}}{}{}
\sphinxAtStartPar
Returns the list of disrupted components.
\begin{quote}\begin{description}
\item[{Returns}] \leavevmode
\sphinxAtStartPar
current list of disrupted components.

\item[{Return type}] \leavevmode
\sphinxAtStartPar
list of strings

\end{description}\end{quote}

\end{fulllineitems}

\index{get\_disrupted\_infra\_dict() (IntegratedNetwork method)@\spxentry{get\_disrupted\_infra\_dict()}\spxextra{IntegratedNetwork method}}

\begin{fulllineitems}
\phantomsection\label{\detokenize{apidoc:dreaminsg_integrated_model.src.network_sim_models.integrated_network.IntegratedNetwork.get_disrupted_infra_dict}}\pysiglinewithargsret{\sphinxbfcode{\sphinxupquote{get\_disrupted\_infra\_dict}}}{}{}
\sphinxAtStartPar
Returns the  disrupted infrastructure components dictionary.
\begin{quote}\begin{description}
\item[{Returns}] \leavevmode
\sphinxAtStartPar
The disrupted infrastructure components dictionary.

\item[{Return type}] \leavevmode
\sphinxAtStartPar
dictionary

\end{description}\end{quote}

\end{fulllineitems}

\index{get\_disruptive\_events() (IntegratedNetwork method)@\spxentry{get\_disruptive\_events()}\spxextra{IntegratedNetwork method}}

\begin{fulllineitems}
\phantomsection\label{\detokenize{apidoc:dreaminsg_integrated_model.src.network_sim_models.integrated_network.IntegratedNetwork.get_disruptive_events}}\pysiglinewithargsret{\sphinxbfcode{\sphinxupquote{get\_disruptive\_events}}}{}{}
\sphinxAtStartPar
Returns the disruptive event data
\begin{description}
\item[{Returns:}] \leavevmode
\sphinxAtStartPar
pandas dataframe: The table with details of disrupted components and the respective damage levels.

\end{description}

\end{fulllineitems}

\index{get\_power\_crew\_loc() (IntegratedNetwork method)@\spxentry{get\_power\_crew\_loc()}\spxextra{IntegratedNetwork method}}

\begin{fulllineitems}
\phantomsection\label{\detokenize{apidoc:dreaminsg_integrated_model.src.network_sim_models.integrated_network.IntegratedNetwork.get_power_crew_loc}}\pysiglinewithargsret{\sphinxbfcode{\sphinxupquote{get\_power\_crew\_loc}}}{}{}
\sphinxAtStartPar
Returns the current power crew location.
\begin{quote}\begin{description}
\item[{Returns}] \leavevmode
\sphinxAtStartPar
Power crew location

\item[{Return type}] \leavevmode
\sphinxAtStartPar
string

\end{description}\end{quote}

\end{fulllineitems}

\index{get\_transpo\_crew\_loc() (IntegratedNetwork method)@\spxentry{get\_transpo\_crew\_loc()}\spxextra{IntegratedNetwork method}}

\begin{fulllineitems}
\phantomsection\label{\detokenize{apidoc:dreaminsg_integrated_model.src.network_sim_models.integrated_network.IntegratedNetwork.get_transpo_crew_loc}}\pysiglinewithargsret{\sphinxbfcode{\sphinxupquote{get\_transpo\_crew\_loc}}}{}{}
\sphinxAtStartPar
Returns the current transportation crew location.
\begin{quote}\begin{description}
\item[{Returns}] \leavevmode
\sphinxAtStartPar
Transportation crew location

\item[{Return type}] \leavevmode
\sphinxAtStartPar
string

\end{description}\end{quote}

\end{fulllineitems}

\index{get\_water\_crew\_loc() (IntegratedNetwork method)@\spxentry{get\_water\_crew\_loc()}\spxextra{IntegratedNetwork method}}

\begin{fulllineitems}
\phantomsection\label{\detokenize{apidoc:dreaminsg_integrated_model.src.network_sim_models.integrated_network.IntegratedNetwork.get_water_crew_loc}}\pysiglinewithargsret{\sphinxbfcode{\sphinxupquote{get\_water\_crew\_loc}}}{}{}
\sphinxAtStartPar
Returns the current water crew location.
\begin{quote}\begin{description}
\item[{Returns}] \leavevmode
\sphinxAtStartPar
Water crew location

\item[{Return type}] \leavevmode
\sphinxAtStartPar
string

\end{description}\end{quote}

\end{fulllineitems}

\index{load\_networks() (IntegratedNetwork method)@\spxentry{load\_networks()}\spxextra{IntegratedNetwork method}}

\begin{fulllineitems}
\phantomsection\label{\detokenize{apidoc:dreaminsg_integrated_model.src.network_sim_models.integrated_network.IntegratedNetwork.load_networks}}\pysiglinewithargsret{\sphinxbfcode{\sphinxupquote{load\_networks}}}{\emph{\DUrole{n}{water\_file}}, \emph{\DUrole{n}{power\_file}}, \emph{\DUrole{n}{transp\_folder}}}{}
\sphinxAtStartPar
Loads the water, power and transportation networks.
\begin{quote}\begin{description}
\item[{Parameters}] \leavevmode\begin{itemize}
\item {} 
\sphinxAtStartPar
\sphinxstyleliteralstrong{\sphinxupquote{water\_file}} (\sphinxstyleliteralemphasis{\sphinxupquote{string}}) \textendash{} The water network file ({\color{red}\bfseries{}*}.inp).

\item {} 
\sphinxAtStartPar
\sphinxstyleliteralstrong{\sphinxupquote{power\_file}} (\sphinxstyleliteralemphasis{\sphinxupquote{string}}) \textendash{} The power systems file ({\color{red}\bfseries{}*}.json).

\item {} 
\sphinxAtStartPar
\sphinxstyleliteralstrong{\sphinxupquote{transp\_folder}} (\sphinxstyleliteralemphasis{\sphinxupquote{string}}) \textendash{} The local directory that consists of required transportation network files.

\end{itemize}

\end{description}\end{quote}

\end{fulllineitems}

\index{pipe\_leak\_node\_generator() (IntegratedNetwork method)@\spxentry{pipe\_leak\_node\_generator()}\spxextra{IntegratedNetwork method}}

\begin{fulllineitems}
\phantomsection\label{\detokenize{apidoc:dreaminsg_integrated_model.src.network_sim_models.integrated_network.IntegratedNetwork.pipe_leak_node_generator}}\pysiglinewithargsret{\sphinxbfcode{\sphinxupquote{pipe\_leak\_node\_generator}}}{}{}
\sphinxAtStartPar
Splits the directly affected pipes to induce leak during simulations.

\end{fulllineitems}

\index{reset\_crew\_locs() (IntegratedNetwork method)@\spxentry{reset\_crew\_locs()}\spxextra{IntegratedNetwork method}}

\begin{fulllineitems}
\phantomsection\label{\detokenize{apidoc:dreaminsg_integrated_model.src.network_sim_models.integrated_network.IntegratedNetwork.reset_crew_locs}}\pysiglinewithargsret{\sphinxbfcode{\sphinxupquote{reset\_crew\_locs}}}{}{}
\sphinxAtStartPar
Resets the location of infrastructure crews.

\end{fulllineitems}

\index{set\_disrupted\_components() (IntegratedNetwork method)@\spxentry{set\_disrupted\_components()}\spxextra{IntegratedNetwork method}}

\begin{fulllineitems}
\phantomsection\label{\detokenize{apidoc:dreaminsg_integrated_model.src.network_sim_models.integrated_network.IntegratedNetwork.set_disrupted_components}}\pysiglinewithargsret{\sphinxbfcode{\sphinxupquote{set\_disrupted\_components}}}{\emph{\DUrole{n}{scenario\_file}}}{}
\sphinxAtStartPar
Sets the disrupted components in the network.
\begin{quote}\begin{description}
\item[{Parameters}] \leavevmode
\sphinxAtStartPar
\sphinxstyleliteralstrong{\sphinxupquote{scenario\_file}} (\sphinxstyleliteralemphasis{\sphinxupquote{string}}) \textendash{} The location of the disruption scenario file in the list.

\end{description}\end{quote}

\end{fulllineitems}

\index{set\_disrupted\_infra\_dict() (IntegratedNetwork method)@\spxentry{set\_disrupted\_infra\_dict()}\spxextra{IntegratedNetwork method}}

\begin{fulllineitems}
\phantomsection\label{\detokenize{apidoc:dreaminsg_integrated_model.src.network_sim_models.integrated_network.IntegratedNetwork.set_disrupted_infra_dict}}\pysiglinewithargsret{\sphinxbfcode{\sphinxupquote{set\_disrupted\_infra\_dict}}}{}{}
\sphinxAtStartPar
Sets the disrupted infrastructure components dictionary with infrastructure type as keys.

\end{fulllineitems}

\index{set\_init\_crew\_locs() (IntegratedNetwork method)@\spxentry{set\_init\_crew\_locs()}\spxextra{IntegratedNetwork method}}

\begin{fulllineitems}
\phantomsection\label{\detokenize{apidoc:dreaminsg_integrated_model.src.network_sim_models.integrated_network.IntegratedNetwork.set_init_crew_locs}}\pysiglinewithargsret{\sphinxbfcode{\sphinxupquote{set\_init\_crew\_locs}}}{\emph{\DUrole{n}{init\_power\_loc}}, \emph{\DUrole{n}{init\_water\_loc}}, \emph{\DUrole{n}{init\_transpo\_loc}}}{}
\sphinxAtStartPar
Sets the intial location of the infrastructure crews. Assign the locations of the respective offices.
\begin{quote}\begin{description}
\item[{Parameters}] \leavevmode\begin{itemize}
\item {} 
\sphinxAtStartPar
\sphinxstyleliteralstrong{\sphinxupquote{init\_power\_loc}} (\sphinxstyleliteralemphasis{\sphinxupquote{string}}) \textendash{} Location (node) of the power crew office.

\item {} 
\sphinxAtStartPar
\sphinxstyleliteralstrong{\sphinxupquote{init\_water\_loc}} (\sphinxstyleliteralemphasis{\sphinxupquote{string}}) \textendash{} Location (node) of the water crew office.

\item {} 
\sphinxAtStartPar
\sphinxstyleliteralstrong{\sphinxupquote{init\_transpo\_loc}} (\sphinxstyleliteralemphasis{\sphinxupquote{string}}) \textendash{} Location (node) of the transportation crew office.

\end{itemize}

\end{description}\end{quote}

\end{fulllineitems}

\index{set\_power\_crew\_loc() (IntegratedNetwork method)@\spxentry{set\_power\_crew\_loc()}\spxextra{IntegratedNetwork method}}

\begin{fulllineitems}
\phantomsection\label{\detokenize{apidoc:dreaminsg_integrated_model.src.network_sim_models.integrated_network.IntegratedNetwork.set_power_crew_loc}}\pysiglinewithargsret{\sphinxbfcode{\sphinxupquote{set\_power\_crew\_loc}}}{\emph{\DUrole{n}{power\_crew\_loc}}}{}
\sphinxAtStartPar
Sets the location of the power crew.
\begin{quote}\begin{description}
\item[{Parameters}] \leavevmode
\sphinxAtStartPar
\sphinxstyleliteralstrong{\sphinxupquote{power\_crew\_loc}} (\sphinxstyleliteralemphasis{\sphinxupquote{string}}) \textendash{} The name of the location (transportation  node)

\end{description}\end{quote}

\end{fulllineitems}

\index{set\_transpo\_crew\_loc() (IntegratedNetwork method)@\spxentry{set\_transpo\_crew\_loc()}\spxextra{IntegratedNetwork method}}

\begin{fulllineitems}
\phantomsection\label{\detokenize{apidoc:dreaminsg_integrated_model.src.network_sim_models.integrated_network.IntegratedNetwork.set_transpo_crew_loc}}\pysiglinewithargsret{\sphinxbfcode{\sphinxupquote{set\_transpo\_crew\_loc}}}{\emph{\DUrole{n}{transpo\_crew\_loc}}}{}
\sphinxAtStartPar
Sets the location of the transportation crew.
\begin{quote}\begin{description}
\item[{Parameters}] \leavevmode
\sphinxAtStartPar
\sphinxstyleliteralstrong{\sphinxupquote{transpo\_crew\_loc}} (\sphinxstyleliteralemphasis{\sphinxupquote{string}}) \textendash{} The name of the location (transportation  node)

\end{description}\end{quote}

\end{fulllineitems}

\index{set\_water\_crew\_loc() (IntegratedNetwork method)@\spxentry{set\_water\_crew\_loc()}\spxextra{IntegratedNetwork method}}

\begin{fulllineitems}
\phantomsection\label{\detokenize{apidoc:dreaminsg_integrated_model.src.network_sim_models.integrated_network.IntegratedNetwork.set_water_crew_loc}}\pysiglinewithargsret{\sphinxbfcode{\sphinxupquote{set\_water\_crew\_loc}}}{\emph{\DUrole{n}{water\_crew\_loc}}}{}
\sphinxAtStartPar
Sets the location of the water crew.
\begin{quote}\begin{description}
\item[{Parameters}] \leavevmode
\sphinxAtStartPar
\sphinxstyleliteralstrong{\sphinxupquote{water\_crew\_loc}} (\sphinxstyleliteralemphasis{\sphinxupquote{string}}) \textendash{} The name of the location (transportation  node)

\end{description}\end{quote}

\end{fulllineitems}


\end{fulllineitems}

\index{Network (class in dreaminsg\_integrated\_model.src.network\_sim\_models.integrated\_network)@\spxentry{Network}\spxextra{class in dreaminsg\_integrated\_model.src.network\_sim\_models.integrated\_network}}

\begin{fulllineitems}
\phantomsection\label{\detokenize{apidoc:dreaminsg_integrated_model.src.network_sim_models.integrated_network.Network}}\pysigline{\sphinxbfcode{\sphinxupquote{class }}\sphinxbfcode{\sphinxupquote{Network}}}
\sphinxAtStartPar
Bases: \sphinxcode{\sphinxupquote{abc.ABC}}

\sphinxAtStartPar
This is an abstract class of integrated infrastructure network, defining an interface to other code. This interface needs to be implemented accordingly.

\end{fulllineitems}



\subsection{dreaminsg\_integrated\_model.src.network\_sim\_models.interdependencies}
\label{\detokenize{apidoc:dreaminsg-integrated-model-src-network-sim-models-interdependencies}}\phantomsection\label{\detokenize{apidoc:module-dreaminsg_integrated_model.src.network_sim_models.interdependencies}}\index{module@\spxentry{module}!dreaminsg\_integrated\_model.src.network\_sim\_models.interdependencies@\spxentry{dreaminsg\_integrated\_model.src.network\_sim\_models.interdependencies}}\index{dreaminsg\_integrated\_model.src.network\_sim\_models.interdependencies@\spxentry{dreaminsg\_integrated\_model.src.network\_sim\_models.interdependencies}!module@\spxentry{module}}
\sphinxAtStartPar
Classes and functions to manage dependencies in the integrated infrastructure network.
\index{DependencyTable (class in dreaminsg\_integrated\_model.src.network\_sim\_models.interdependencies)@\spxentry{DependencyTable}\spxextra{class in dreaminsg\_integrated\_model.src.network\_sim\_models.interdependencies}}

\begin{fulllineitems}
\phantomsection\label{\detokenize{apidoc:dreaminsg_integrated_model.src.network_sim_models.interdependencies.DependencyTable}}\pysigline{\sphinxbfcode{\sphinxupquote{class }}\sphinxbfcode{\sphinxupquote{DependencyTable}}}
\sphinxAtStartPar
Bases: \sphinxcode{\sphinxupquote{object}}

\sphinxAtStartPar
A class to store information related to dependencies among power, water and transportation networks.
\index{add\_gen\_reserv\_coupling() (DependencyTable method)@\spxentry{add\_gen\_reserv\_coupling()}\spxextra{DependencyTable method}}

\begin{fulllineitems}
\phantomsection\label{\detokenize{apidoc:dreaminsg_integrated_model.src.network_sim_models.interdependencies.DependencyTable.add_gen_reserv_coupling}}\pysiglinewithargsret{\sphinxbfcode{\sphinxupquote{add\_gen\_reserv\_coupling}}}{\emph{\DUrole{n}{water\_id}}, \emph{\DUrole{n}{power\_id}}}{}
\sphinxAtStartPar
Creates a generator\sphinxhyphen{}on\sphinxhyphen{}reservoir dependency entry in the dependency table.
\begin{quote}\begin{description}
\item[{Parameters}] \leavevmode\begin{itemize}
\item {} 
\sphinxAtStartPar
\sphinxstyleliteralstrong{\sphinxupquote{water\_id}} (\sphinxstyleliteralemphasis{\sphinxupquote{string}}) \textendash{} The name of the reservoir in the water network model.

\item {} 
\sphinxAtStartPar
\sphinxstyleliteralstrong{\sphinxupquote{power\_id}} (\sphinxstyleliteralemphasis{\sphinxupquote{string}}) \textendash{} The name of the generator in the power systems model.

\end{itemize}

\end{description}\end{quote}

\end{fulllineitems}

\index{add\_pump\_motor\_coupling() (DependencyTable method)@\spxentry{add\_pump\_motor\_coupling()}\spxextra{DependencyTable method}}

\begin{fulllineitems}
\phantomsection\label{\detokenize{apidoc:dreaminsg_integrated_model.src.network_sim_models.interdependencies.DependencyTable.add_pump_motor_coupling}}\pysiglinewithargsret{\sphinxbfcode{\sphinxupquote{add\_pump\_motor\_coupling}}}{\emph{\DUrole{n}{water\_id}}, \emph{\DUrole{n}{power\_id}}}{}
\sphinxAtStartPar
Creates a pump\sphinxhyphen{}on\sphinxhyphen{}motor dependency entry in the dependency table.
\begin{quote}\begin{description}
\item[{Parameters}] \leavevmode\begin{itemize}
\item {} 
\sphinxAtStartPar
\sphinxstyleliteralstrong{\sphinxupquote{water\_id}} (\sphinxstyleliteralemphasis{\sphinxupquote{string}}) \textendash{} The name of the pump in the water network model.

\item {} 
\sphinxAtStartPar
\sphinxstyleliteralstrong{\sphinxupquote{power\_id}} (\sphinxstyleliteralemphasis{\sphinxupquote{string}}) \textendash{} The name of the motor in the power systems model.

\end{itemize}

\end{description}\end{quote}

\end{fulllineitems}

\index{add\_transpo\_access() (DependencyTable method)@\spxentry{add\_transpo\_access()}\spxextra{DependencyTable method}}

\begin{fulllineitems}
\phantomsection\label{\detokenize{apidoc:dreaminsg_integrated_model.src.network_sim_models.interdependencies.DependencyTable.add_transpo_access}}\pysiglinewithargsret{\sphinxbfcode{\sphinxupquote{add\_transpo\_access}}}{\emph{\DUrole{n}{integrated\_graph}}}{}
\sphinxAtStartPar
Creates a mapping to nearest road link from every water/power network component.
\begin{quote}\begin{description}
\item[{Parameters}] \leavevmode
\sphinxAtStartPar
\sphinxstyleliteralstrong{\sphinxupquote{integrated\_graph}} (\sphinxstyleliteralemphasis{\sphinxupquote{{[}}}\sphinxstyleliteralemphasis{\sphinxupquote{networkx object}}\sphinxstyleliteralemphasis{\sphinxupquote{{]}}}) \textendash{} The integrated network as networkx object.

\end{description}\end{quote}

\end{fulllineitems}

\index{build\_power\_water\_dependencies() (DependencyTable method)@\spxentry{build\_power\_water\_dependencies()}\spxextra{DependencyTable method}}

\begin{fulllineitems}
\phantomsection\label{\detokenize{apidoc:dreaminsg_integrated_model.src.network_sim_models.interdependencies.DependencyTable.build_power_water_dependencies}}\pysiglinewithargsret{\sphinxbfcode{\sphinxupquote{build\_power\_water\_dependencies}}}{\emph{\DUrole{n}{dependency\_file}}}{}
\sphinxAtStartPar
Adds the power\sphinxhyphen{}water dependency table to the DependencyTable object.
\begin{quote}\begin{description}
\item[{Parameters}] \leavevmode
\sphinxAtStartPar
\sphinxstyleliteralstrong{\sphinxupquote{dependency\_file}} (\sphinxstyleliteralemphasis{\sphinxupquote{string}}) \textendash{} The location of the dependency file containing dependency information.

\end{description}\end{quote}

\end{fulllineitems}

\index{build\_transportation\_access() (DependencyTable method)@\spxentry{build\_transportation\_access()}\spxextra{DependencyTable method}}

\begin{fulllineitems}
\phantomsection\label{\detokenize{apidoc:dreaminsg_integrated_model.src.network_sim_models.interdependencies.DependencyTable.build_transportation_access}}\pysiglinewithargsret{\sphinxbfcode{\sphinxupquote{build\_transportation\_access}}}{\emph{\DUrole{n}{integrated\_graph}}}{}
\sphinxAtStartPar
Adds the transportatio naccess table to the DependencyTable object.
\begin{quote}\begin{description}
\item[{Parameters}] \leavevmode
\sphinxAtStartPar
\sphinxstyleliteralstrong{\sphinxupquote{integrated\_graph}} (\sphinxstyleliteralemphasis{\sphinxupquote{Nextworkx object}}) \textendash{} The integrated network as Networkx object.

\end{description}\end{quote}

\end{fulllineitems}

\index{update\_dependencies() (DependencyTable method)@\spxentry{update\_dependencies()}\spxextra{DependencyTable method}}

\begin{fulllineitems}
\phantomsection\label{\detokenize{apidoc:dreaminsg_integrated_model.src.network_sim_models.interdependencies.DependencyTable.update_dependencies}}\pysiglinewithargsret{\sphinxbfcode{\sphinxupquote{update\_dependencies}}}{\emph{\DUrole{n}{network}}, \emph{\DUrole{n}{time\_stamp}}, \emph{\DUrole{n}{next\_time\_stamp}}}{}
\sphinxAtStartPar
Updates the operational performance of all the dependent components in the integrated network.
\begin{quote}\begin{description}
\item[{Parameters}] \leavevmode\begin{itemize}
\item {} 
\sphinxAtStartPar
\sphinxstyleliteralstrong{\sphinxupquote{network}} (\sphinxstyleliteralemphasis{\sphinxupquote{An IntegratedNetwork object}}) \textendash{} The integrated infrastructure network object.

\item {} 
\sphinxAtStartPar
\sphinxstyleliteralstrong{\sphinxupquote{time\_stamp}} (\sphinxstyleliteralemphasis{\sphinxupquote{integer}}) \textendash{} The start time of the current iteration in seconds.

\item {} 
\sphinxAtStartPar
\sphinxstyleliteralstrong{\sphinxupquote{next\_time\_stamp}} (\sphinxstyleliteralemphasis{\sphinxupquote{integer}}) \textendash{} The end tiem of the iteration.

\end{itemize}

\end{description}\end{quote}

\end{fulllineitems}


\end{fulllineitems}

\index{find\_connected\_power\_node() (in module dreaminsg\_integrated\_model.src.network\_sim\_models.interdependencies)@\spxentry{find\_connected\_power\_node()}\spxextra{in module dreaminsg\_integrated\_model.src.network\_sim\_models.interdependencies}}

\begin{fulllineitems}
\phantomsection\label{\detokenize{apidoc:dreaminsg_integrated_model.src.network_sim_models.interdependencies.find_connected_power_node}}\pysiglinewithargsret{\sphinxbfcode{\sphinxupquote{find\_connected\_power\_node}}}{\emph{\DUrole{n}{component}}, \emph{\DUrole{n}{pn}}}{}
\sphinxAtStartPar
Finds the bus to which the given power systems component is connected to. For elements which are connected to two buses, the start bus is returned.
\begin{quote}\begin{description}
\item[{Parameters}] \leavevmode\begin{itemize}
\item {} 
\sphinxAtStartPar
\sphinxstyleliteralstrong{\sphinxupquote{component}} (\sphinxstyleliteralemphasis{\sphinxupquote{string}}) \textendash{} Name of the power systems component.

\item {} 
\sphinxAtStartPar
\sphinxstyleliteralstrong{\sphinxupquote{pn}} (\sphinxstyleliteralemphasis{\sphinxupquote{pandapower network object}}) \textendash{} The power network the origin node belongs to.

\end{itemize}

\item[{Returns}] \leavevmode
\sphinxAtStartPar
Name of the connected bus.

\item[{Return type}] \leavevmode
\sphinxAtStartPar
string

\end{description}\end{quote}

\end{fulllineitems}

\index{find\_connected\_transpo\_node() (in module dreaminsg\_integrated\_model.src.network\_sim\_models.interdependencies)@\spxentry{find\_connected\_transpo\_node()}\spxextra{in module dreaminsg\_integrated\_model.src.network\_sim\_models.interdependencies}}

\begin{fulllineitems}
\phantomsection\label{\detokenize{apidoc:dreaminsg_integrated_model.src.network_sim_models.interdependencies.find_connected_transpo_node}}\pysiglinewithargsret{\sphinxbfcode{\sphinxupquote{find\_connected\_transpo\_node}}}{\emph{\DUrole{n}{component}}, \emph{\DUrole{n}{tn}}}{}
\sphinxAtStartPar
Finds the bus to which the given power systems component is connected to. For elements which are connected to two buses, the start bus is returned.
\begin{quote}\begin{description}
\item[{Parameters}] \leavevmode\begin{itemize}
\item {} 
\sphinxAtStartPar
\sphinxstyleliteralstrong{\sphinxupquote{component}} (\sphinxstyleliteralemphasis{\sphinxupquote{string}}) \textendash{} Name of the power systems component.

\item {} 
\sphinxAtStartPar
\sphinxstyleliteralstrong{\sphinxupquote{pn}} (\sphinxstyleliteralemphasis{\sphinxupquote{pandapower network object}}) \textendash{} The power network the origin node belongs to.

\end{itemize}

\item[{Returns}] \leavevmode
\sphinxAtStartPar
Name of the connected bus.

\item[{Return type}] \leavevmode
\sphinxAtStartPar
string

\end{description}\end{quote}

\end{fulllineitems}

\index{find\_connected\_water\_node() (in module dreaminsg\_integrated\_model.src.network\_sim\_models.interdependencies)@\spxentry{find\_connected\_water\_node()}\spxextra{in module dreaminsg\_integrated\_model.src.network\_sim\_models.interdependencies}}

\begin{fulllineitems}
\phantomsection\label{\detokenize{apidoc:dreaminsg_integrated_model.src.network_sim_models.interdependencies.find_connected_water_node}}\pysiglinewithargsret{\sphinxbfcode{\sphinxupquote{find\_connected\_water\_node}}}{\emph{\DUrole{n}{component}}, \emph{\DUrole{n}{wn}}}{}
\sphinxAtStartPar
Finds the water network node to which the water component is connected to.
\begin{quote}\begin{description}
\item[{Parameters}] \leavevmode\begin{itemize}
\item {} 
\sphinxAtStartPar
\sphinxstyleliteralstrong{\sphinxupquote{component}} (\sphinxstyleliteralemphasis{\sphinxupquote{string}}) \textendash{} Name of the water network component.

\item {} 
\sphinxAtStartPar
\sphinxstyleliteralstrong{\sphinxupquote{wn}} (\sphinxstyleliteralemphasis{\sphinxupquote{wntr network object}}) \textendash{} The water distribution network the origin node belongs to.

\end{itemize}

\item[{Returns}] \leavevmode
\sphinxAtStartPar
Name of the water network node.

\item[{Return type}] \leavevmode
\sphinxAtStartPar
string

\end{description}\end{quote}

\end{fulllineitems}

\index{get\_compon\_details() (in module dreaminsg\_integrated\_model.src.network\_sim\_models.interdependencies)@\spxentry{get\_compon\_details()}\spxextra{in module dreaminsg\_integrated\_model.src.network\_sim\_models.interdependencies}}

\begin{fulllineitems}
\phantomsection\label{\detokenize{apidoc:dreaminsg_integrated_model.src.network_sim_models.interdependencies.get_compon_details}}\pysiglinewithargsret{\sphinxbfcode{\sphinxupquote{get\_compon\_details}}}{\emph{\DUrole{n}{compon\_name}}}{}
\sphinxAtStartPar
Fetches the infrastructure type, component type, component code and component actual name.
\begin{quote}\begin{description}
\item[{Parameters}] \leavevmode
\sphinxAtStartPar
\sphinxstyleliteralstrong{\sphinxupquote{compon\_name}} (\sphinxstyleliteralemphasis{\sphinxupquote{string}}) \textendash{} Name of the component.

\item[{Returns}] \leavevmode
\sphinxAtStartPar
Infrastructure type, component type, component code and component actual name.

\item[{Return type}] \leavevmode
\sphinxAtStartPar
list of strings

\end{description}\end{quote}

\end{fulllineitems}

\index{get\_nearest\_node() (in module dreaminsg\_integrated\_model.src.network\_sim\_models.interdependencies)@\spxentry{get\_nearest\_node()}\spxextra{in module dreaminsg\_integrated\_model.src.network\_sim\_models.interdependencies}}

\begin{fulllineitems}
\phantomsection\label{\detokenize{apidoc:dreaminsg_integrated_model.src.network_sim_models.interdependencies.get_nearest_node}}\pysiglinewithargsret{\sphinxbfcode{\sphinxupquote{get\_nearest\_node}}}{\emph{\DUrole{n}{integrated\_graph}}, \emph{\DUrole{n}{connected\_node}}, \emph{\DUrole{n}{target\_type}}}{}
\sphinxAtStartPar
Finds the nearest node belonging to a specific family from a given node and the distance between the two.
\begin{quote}\begin{description}
\item[{Parameters}] \leavevmode\begin{itemize}
\item {} 
\sphinxAtStartPar
\sphinxstyleliteralstrong{\sphinxupquote{integrated\_graph}} (\sphinxstyleliteralemphasis{\sphinxupquote{netwrokx object}}) \textendash{} The integrated network in networkx format.

\item {} 
\sphinxAtStartPar
\sphinxstyleliteralstrong{\sphinxupquote{connected\_node}} (\sphinxstyleliteralemphasis{\sphinxupquote{string/integer}}) \textendash{} Name of the node for which the nearest node has to be identified.

\item {} 
\sphinxAtStartPar
\sphinxstyleliteralstrong{\sphinxupquote{target\_type}} (\sphinxstyleliteralemphasis{\sphinxupquote{string}}) \textendash{} The type of the target node (power\_node, transpo\_node, water\_node)

\end{itemize}

\item[{Returns}] \leavevmode
\sphinxAtStartPar
Nearest node belonging to target type and the distance in meters.

\item[{Return type}] \leavevmode
\sphinxAtStartPar
list

\end{description}\end{quote}

\end{fulllineitems}



\subsection{dreaminsg\_integrated\_model.src.network\_sim\_models.power.power\_system\_model}
\label{\detokenize{apidoc:dreaminsg-integrated-model-src-network-sim-models-power-power-system-model}}\phantomsection\label{\detokenize{apidoc:module-dreaminsg_integrated_model.src.network_sim_models.power.power_system_model}}\index{module@\spxentry{module}!dreaminsg\_integrated\_model.src.network\_sim\_models.power.power\_system\_model@\spxentry{dreaminsg\_integrated\_model.src.network\_sim\_models.power.power\_system\_model}}\index{dreaminsg\_integrated\_model.src.network\_sim\_models.power.power\_system\_model@\spxentry{dreaminsg\_integrated\_model.src.network\_sim\_models.power.power\_system\_model}!module@\spxentry{module}}
\sphinxAtStartPar
Functions to implement power systems simulations.
\index{get\_power\_dict() (in module dreaminsg\_integrated\_model.src.network\_sim\_models.power.power\_system\_model)@\spxentry{get\_power\_dict()}\spxextra{in module dreaminsg\_integrated\_model.src.network\_sim\_models.power.power\_system\_model}}

\begin{fulllineitems}
\phantomsection\label{\detokenize{apidoc:dreaminsg_integrated_model.src.network_sim_models.power.power_system_model.get_power_dict}}\pysiglinewithargsret{\sphinxbfcode{\sphinxupquote{get\_power\_dict}}}{}{}
\sphinxAtStartPar
Creates a dictionary of major power system components in a network. Used for naming automatically generated networks.
\begin{quote}\begin{description}
\item[{Returns}] \leavevmode
\sphinxAtStartPar
Mapping of infrastructure component abbreviations to names.

\item[{Return type}] \leavevmode
\sphinxAtStartPar
dictionary of string: dictionary of string: string

\end{description}\end{quote}

\end{fulllineitems}

\index{load\_power\_network() (in module dreaminsg\_integrated\_model.src.network\_sim\_models.power.power\_system\_model)@\spxentry{load\_power\_network()}\spxextra{in module dreaminsg\_integrated\_model.src.network\_sim\_models.power.power\_system\_model}}

\begin{fulllineitems}
\phantomsection\label{\detokenize{apidoc:dreaminsg_integrated_model.src.network_sim_models.power.power_system_model.load_power_network}}\pysiglinewithargsret{\sphinxbfcode{\sphinxupquote{load\_power\_network}}}{\emph{\DUrole{n}{network\_json}}}{}
\sphinxAtStartPar
Loads the power system model from a {\color{red}\bfseries{}*}.json file.
\begin{quote}\begin{description}
\item[{Parameters}] \leavevmode
\sphinxAtStartPar
\sphinxstyleliteralstrong{\sphinxupquote{network\_json}} (\sphinxstyleliteralemphasis{\sphinxupquote{string}}) \textendash{} Location of the {\color{red}\bfseries{}*}.json power system file generated by pandapower package.

\item[{Returns}] \leavevmode
\sphinxAtStartPar
The loaded power system model object.

\item[{Return type}] \leavevmode
\sphinxAtStartPar
pandapower network object

\end{description}\end{quote}

\end{fulllineitems}

\index{run\_power\_simulation() (in module dreaminsg\_integrated\_model.src.network\_sim\_models.power.power\_system\_model)@\spxentry{run\_power\_simulation()}\spxextra{in module dreaminsg\_integrated\_model.src.network\_sim\_models.power.power\_system\_model}}

\begin{fulllineitems}
\phantomsection\label{\detokenize{apidoc:dreaminsg_integrated_model.src.network_sim_models.power.power_system_model.run_power_simulation}}\pysiglinewithargsret{\sphinxbfcode{\sphinxupquote{run\_power\_simulation}}}{\emph{\DUrole{n}{pn}}}{}
\sphinxAtStartPar
Runs the power flow model for an instance.
\begin{quote}\begin{description}
\item[{Parameters}] \leavevmode
\sphinxAtStartPar
\sphinxstyleliteralstrong{\sphinxupquote{pn}} (\sphinxstyleliteralemphasis{\sphinxupquote{pandapower network object}}) \textendash{} A power system model object generated by pandapower package.

\end{description}\end{quote}

\end{fulllineitems}



\subsection{dreaminsg\_integrated\_model.src.network\_sim\_models.water.water\_network\_model}
\label{\detokenize{apidoc:dreaminsg-integrated-model-src-network-sim-models-water-water-network-model}}\phantomsection\label{\detokenize{apidoc:module-dreaminsg_integrated_model.src.network_sim_models.water.water_network_model}}\index{module@\spxentry{module}!dreaminsg\_integrated\_model.src.network\_sim\_models.water.water\_network\_model@\spxentry{dreaminsg\_integrated\_model.src.network\_sim\_models.water.water\_network\_model}}\index{dreaminsg\_integrated\_model.src.network\_sim\_models.water.water\_network\_model@\spxentry{dreaminsg\_integrated\_model.src.network\_sim\_models.water.water\_network\_model}!module@\spxentry{module}}
\sphinxAtStartPar
Functions to implement water network simulations.
\index{get\_water\_dict() (in module dreaminsg\_integrated\_model.src.network\_sim\_models.water.water\_network\_model)@\spxentry{get\_water\_dict()}\spxextra{in module dreaminsg\_integrated\_model.src.network\_sim\_models.water.water\_network\_model}}

\begin{fulllineitems}
\phantomsection\label{\detokenize{apidoc:dreaminsg_integrated_model.src.network_sim_models.water.water_network_model.get_water_dict}}\pysiglinewithargsret{\sphinxbfcode{\sphinxupquote{get\_water\_dict}}}{}{}
\sphinxAtStartPar
Creates a dictionary of major water distribution system components. Used for naming automatically generated networks.
\begin{quote}\begin{description}
\item[{Returns}] \leavevmode
\sphinxAtStartPar
Mapping of infrastructure component abbreviations to names.

\item[{Return type}] \leavevmode
\sphinxAtStartPar
dictionary of string: string

\end{description}\end{quote}

\end{fulllineitems}

\index{load\_water\_network() (in module dreaminsg\_integrated\_model.src.network\_sim\_models.water.water\_network\_model)@\spxentry{load\_water\_network()}\spxextra{in module dreaminsg\_integrated\_model.src.network\_sim\_models.water.water\_network\_model}}

\begin{fulllineitems}
\phantomsection\label{\detokenize{apidoc:dreaminsg_integrated_model.src.network_sim_models.water.water_network_model.load_water_network}}\pysiglinewithargsret{\sphinxbfcode{\sphinxupquote{load\_water\_network}}}{\emph{\DUrole{n}{network\_inp}}, \emph{\DUrole{n}{initial\_sim\_step}}}{}
\sphinxAtStartPar
Loads the water network model from an {\color{red}\bfseries{}*}.inp file.
\begin{quote}\begin{description}
\item[{Parameters}] \leavevmode\begin{itemize}
\item {} 
\sphinxAtStartPar
\sphinxstyleliteralstrong{\sphinxupquote{network\_inp}} (\sphinxstyleliteralemphasis{\sphinxupquote{string}}) \textendash{} Location of the {\color{red}\bfseries{}*}.inp water network file.

\item {} 
\sphinxAtStartPar
\sphinxstyleliteralstrong{\sphinxupquote{initial\_sim\_step}} (\sphinxstyleliteralemphasis{\sphinxupquote{integer}}) \textendash{} The initial iteration step size in seconds.

\end{itemize}

\item[{Returns}] \leavevmode
\sphinxAtStartPar
The loaded water wntr network object.

\item[{Return type}] \leavevmode
\sphinxAtStartPar
wntr network object

\end{description}\end{quote}

\end{fulllineitems}

\index{run\_water\_simulation() (in module dreaminsg\_integrated\_model.src.network\_sim\_models.water.water\_network\_model)@\spxentry{run\_water\_simulation()}\spxextra{in module dreaminsg\_integrated\_model.src.network\_sim\_models.water.water\_network\_model}}

\begin{fulllineitems}
\phantomsection\label{\detokenize{apidoc:dreaminsg_integrated_model.src.network_sim_models.water.water_network_model.run_water_simulation}}\pysiglinewithargsret{\sphinxbfcode{\sphinxupquote{run\_water\_simulation}}}{\emph{\DUrole{n}{wn}}}{}
\sphinxAtStartPar
Runs the simulation for one time step.
\begin{quote}\begin{description}
\item[{Parameters}] \leavevmode
\sphinxAtStartPar
\sphinxstyleliteralstrong{\sphinxupquote{wn}} (\sphinxstyleliteralemphasis{\sphinxupquote{{[}}}\sphinxstyleliteralemphasis{\sphinxupquote{type}}\sphinxstyleliteralemphasis{\sphinxupquote{{]}}}) \textendash{} Water network model object.

\item[{Returns}] \leavevmode
\sphinxAtStartPar
Simulation results in pandas tables.

\item[{Return type}] \leavevmode
\sphinxAtStartPar
ordered dictionary of string: pandas table

\end{description}\end{quote}

\end{fulllineitems}



\chapter{SEC Disclaimer}
\label{\detokenize{index:sec-disclaimer}}
\sphinxAtStartPar
This research is carried out by Singapore ETHC Centre through its Future Resilient Systems module
funded by the National Research Foundation (NRF) Singapore. It is subject to Agency’s review and hence is for internal use only.
Not the contents necessarily reflect the views of the Agency. Mention of trade names, products, or services does not convey official
NRF approval, endorsement, or recommendation.


\chapter{Funding Statement}
\label{\detokenize{index:funding-statement}}
\sphinxAtStartPar
The project is funded by the National Research Foundation Singapore through the Inter\sphinxhyphen{}CREATE program.


\section{Indices and tables}
\label{\detokenize{index:indices-and-tables}}
\sphinxAtStartPar
To read the various modules and methods in the package in the alphabetical order, click on Index. To read about the module\sphinxhyphen{}wise
details, click on Module index.
\begin{itemize}
\item {} 
\sphinxAtStartPar
\DUrole{xref,std,std-ref}{genindex}

\item {} 
\sphinxAtStartPar
\DUrole{xref,std,std-ref}{modindex}

\item {} 
\sphinxAtStartPar
\DUrole{xref,std,std-ref}{search}

\end{itemize}


\renewcommand{\indexname}{Python Module Index}
\begin{sphinxtheindex}
\let\bigletter\sphinxstyleindexlettergroup
\bigletter{d}
\item\relax\sphinxstyleindexentry{dreaminsg\_integrated\_model.src.network\_sim\_models.integrated\_network}\sphinxstyleindexpageref{apidoc:\detokenize{module-dreaminsg_integrated_model.src.network_sim_models.integrated_network}}
\item\relax\sphinxstyleindexentry{dreaminsg\_integrated\_model.src.network\_sim\_models.interdependencies}\sphinxstyleindexpageref{apidoc:\detokenize{module-dreaminsg_integrated_model.src.network_sim_models.interdependencies}}
\item\relax\sphinxstyleindexentry{dreaminsg\_integrated\_model.src.network\_sim\_models.power.power\_system\_model}\sphinxstyleindexpageref{apidoc:\detokenize{module-dreaminsg_integrated_model.src.network_sim_models.power.power_system_model}}
\item\relax\sphinxstyleindexentry{dreaminsg\_integrated\_model.src.network\_sim\_models.water.water\_network\_model}\sphinxstyleindexpageref{apidoc:\detokenize{module-dreaminsg_integrated_model.src.network_sim_models.water.water_network_model}}
\item\relax\sphinxstyleindexentry{dreaminsg\_integrated\_model.src.optimizer}\sphinxstyleindexpageref{apidoc:\detokenize{module-dreaminsg_integrated_model.src.optimizer}}
\end{sphinxtheindex}

\renewcommand{\indexname}{Index}
\printindex
\end{document}